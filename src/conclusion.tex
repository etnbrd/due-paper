\section{Conclusion} \label{section:conclusion}

In this paper, we introduce a compiler to automatically transform an imbrication of continuations into a sequence of Due.
First, we define callbacks and Promises, and then introduce Dues, a siùpler specification to Promises.
We explain the transformation from the nested imbrication of continuations to a chain of Dues.
We present a compiler to automate this transformation on code bases, and evaluate it against a set of projects.

% A continuation share its scope with its descendance, \textit{i.e.} the following imbricated continuations.
% Imbricated continuations can share identifiers.
% While a due callback can not share identifiers with the following dues.
% Their scopes are disjoints, still, sequence of dues can share global identifiers and object references.
% This difference of accessibility implies, after compilation, the segmentation of the asynchronous control flow into indepenent steps.
% This segmentation is soft : their stacks are independent, but they share the heap.

This transformation flatten a nested imbrication of continuations.
The result is a sequence of operations encapsulated in Dues.
Dues, like Promises, arrange both the control and data flow.
They are placeholder for the outcome of an operation.
The outcome of an operation is assigned as the input of the next.
Such an arrangement is very suggestive of a data flow process, that is a chain of operations feeding the next with the result of the previous.

% Web applications manipulate flows of user requests.
We aim at pushing further this analogy.
A web application manipulates flow of user requests.
We think it is possible to arrange such an application as a chain of independent operations communicating by flow of messages.
We want to impose the compiler to bring complete independence to the asynchronous operations delimited by the Dues.
% So that the only communication is of their results along the flow.
Such a compiler would be able to transform a monolithic program into a chain of independent asynchronous operations linked by a flow of data.
We expect the possibility for new execution models to take advantage of this independence to bring performance scalability.
While developers would continue using the monolithic model for its evolution scalability.