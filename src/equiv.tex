\section{Equivalence} \label{section:equivalence}

We present the transformation from a nested imbrication of continuations into a chain of Dues.
% The transformation must preserve the semantic.
We explain the three limitations imposed by the compiler for this transformation to preserve the semantic.
The limitations preserves the execution order, the execution linearity and the scopes of the variables used in the operations.

\subsection{Execution order}

The compiler spots function calls with a continuation.
Such calls are similar to the abstraction in (\ref{eq:order:source}).
It relocates the continuation in a call to the method $\textbf{then}$, which references the Due returned by the function $fn_\textbf{due}$, wrapping $fn$.
The result should be similar to (\ref{eq:order:target}).
\begin{equation} \label{eq:order:source}
fn([arguments], continuation)
\end{equation}
\begin{equation} \label{eq:order:target}
fn_\textbf{due}([arguments])\textbf{.then}(continuation)
\end{equation}

The execution order is different whether $continuation$ is called synchronously, or asynchronously.
If $fn$ is synchronous, it calls the $continuation$ within its execution.
% There might be some statements after the execution of the $continuation$.
If $fn$ is asynchronous, the continuation is called after the end of the current execution, after $fn$.
The transformation erases this difference execution order.
In both cases, the transformation relocates the execution of $continuation$ after the execution of $fn$.
$fn$ must be asynchronous to preserve the execution order.

\subsection{Execution linearity}

The compiler transform nested imbrication of continuation, like in (\ref{eq:state:source}) into a flatten chain of calls encapsulating continuations, like in (\ref{eq:state:target}).
\begin{align} \label{eq:state:source}
&fn1([arguments], cont1 \{\nonumber\\
&\qquad  declare ~ variable\nonumber\\
&\qquad  fn2([arguments], cont2 \{\nonumber\\
&\qquad\qquad    print variable\nonumber\\
&\qquad  \})\nonumber\\
&\})
\end{align}
\begin{align} \label{eq:state:target}
&\textbf{declare variable}\nonumber\\
&fn1_\textbf{due}([arguments])\nonumber\\
&\textbf{.then}(cont1\{\nonumber\\
&\qquad  fn2_\textbf{due}([arguments])\nonumber\\
&\})\nonumber\\
&\textbf{.then}(cont2\{\nonumber\\
&\qquad  print ~ variable\nonumber\\
&\})
\end{align}

An imbrication of continuations can not contain a loop, nor a function definition that is not a continuation.
Both modify the linearity of the execution flow which is required for the equivalence to keep the semantic.
A call nested inside a loop returns multiple Dues, while only one is returned to continue the chain.
And a call nested inside a function definition is unable to return the Due to continue the chain.
On the other hand, conditional branching leaves the execution linearity and the semantic intact.
If the nested asynchronous function is not called, the execution of the chain breaks as expected.

\subsection{Variable scope}

In \ref{eq:state:source}, the definitions of $cont1$ and $cont2$ are overlapping.
The $variable$ of $cont1$ are accessible to $cont2$.
In \ref{eq:state:target}, their definitions are not overlapping, they are siblings.
The $variable$ is not accessible to $cont2$, it must be relocated in a parent function to be accessible by both $cont1$ and $cont2$.
To detect such variables, the compiler must infer their scope statically.
It is possible only for lexically scoped languages.
Most imperative languages are lexically scoped.
The subset of Javascript excluding the built-in functions \texttt{with} and \texttt{eval} is also lexically scoped.