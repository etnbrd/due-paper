\section{Definitions} \label{section:definitions}

\subsection{Callbacks} \label{section:definitions:callback}

A callback is a function passed as an argument to another function to defer its execution.
It allows to process arguments not available in the current context.
We distinguish three kinds of callbacks.
\begin{itemize}
  \item An \textbf{Iterator} is a function called for each item in a set.
  % In \textit{node.js}, the methods of the Array prototype expect iterators, \textit{e.g.} \texttt{forEach}, \texttt{map}, \texttt{map}.
  An iterator is often called synchronously.
  \item A \textbf{Listener} is a function called for each message in a stream.
  A listener is called asynchronously, for each new available message.
  \item A \textbf{Continuation} is a function called asynchronously once a unique result is available.
  Callbacks are often mistaken for continuations.
  We only focus on continuation in this paper, because promises - or dues - can replace only continuations.
\end{itemize}

When continuation are used, the convention on how to handle the result must be common for both the caller and the caller.
In \textit{Node.js}, the signature of a continuation uses the convention \textit{error-first}\footnote{\label{ftn:error-first}\url{https://docs.nodejitsu.com/articles/errors/what-are-the-error-conventions}}\footnote{\url{http://programmers.stackexchange.com/questions/144089/different-callbacks-for-error-or-error-as-first-argument}}.
The first argument contains an error or \texttt{null} if no error occurred ; then follows the result.
Listing \ref{lst:continuation} is an example of continuation.
The \texttt{my_fn} function is defined in listing \ref{lst:my-fn}.
Another convention is to provide two distinct functions to handle results and errors. 

\begin{code}[js, %
             caption={Example of a continuation}, %
             label={lst:continuation}] %
my_fn(<arg>, function continuation(error, result) {
  if (!error) {
    // do something with result ...
  } else {
    // do something with error ...
  }
});
\end{code}

% The continuation allows to continue the execution sequentially, after the result of \textit{my_fn} is available. 
% When continuations are defined inside the call, like \textit{continuation}, the sequence of deferred execution results in an intricate imbrication of calls and continuations, like in listing \ref{lst:cbhell}.
The callback hell occurs when many asynchronous calls are cascaded, like in listing \ref{lst:cbhell}.
Each caller must wait for the result before calling the next function.
Promise allows to arrange a sequence of deferred execution in a more readable way.

\begin{code}[js, %
             caption={Example of a cascade of continuations}, %
             label={lst:cbhell}] %
my_fn_1(arg, function continuation(error, result) {
  if (!error) {
    // do something with result to get arg
    my_fn_2(arg, function continuation(error, result) {
      if (!error) {
        // do something with result to get arg
        my_fn_3(arg, function continuation(error, result) {
          if (!error) {
            // do something with result ...
          } else {
            // do something with error ...
          }
        });
      } else {
        // do something with error ...
      }
    });
  } else {
    // do something with error ...
  }
});
\end{code}

\subsection{Promise} \label{section:definitions:promise}

% TODO insert these :
% Promise also provide few methods to enhance the asynchronous control flow tools\footnote{\texttt{all} and \texttt{race}}.
% There is, in Javascript, numerous Promise implementations\footnote{37 different implementations in Javascript \url{https://github.com/promises-aplus/promises-spec/blob/master/implementations.md}}.

This section is based on the Promises section of the specification in ECMAScript 6 Harmony\footnote{\url{https://people.mozilla.org/~jorendorff/es6-draft.html\#sec-promise-objects}} and the Promises page on the Mozilla Developer Network\footnote{\url{https://developer.mozilla.org/en/docs/Web/JavaScript/Reference/Global_Objects/Promise}}.
The specification defines a promise as \textit{an object that is used as a placeholder for the eventual results of a deferred (and possibly asynchronous) computation.} (ECMAScript 6 Harmony Specification, section 25.4 Promise Objects)

A promise is an object returned by a function to represent its result.
Because it is possibly unavailable synchronously, it still requires a callback to defer the execution when the result is made available.
A promise also requires another callback to defer the execution in case of errors.
This two callbacks are passed to the method \texttt{then} of the promise, like illustrated in listing \ref{lst:then}.

\begin{code}[js, %
             caption={Example of a promise}, %
             label={lst:then}] %
var promise = my_fn(<arg>)

promise.then(function onSuccess(result) {
  // do something with result ...
}, function onErrors(reason) {
  // do something with the reason of the error ...
});
\end{code}

A sequence of promises is arranged as a cascade of call, by opposition to the imbrication of callbacks.
To allow cascading, the method \texttt{then} returns a promise which resolves when the promise returned by its callbacks resolve.
This is illustrated in listing \ref{lst:promises-sequence}.
The two first \texttt{onSuccess} callbacks call \texttt{my\_fn\_2} and \texttt{my\_fn\_3}, return the promises \texttt{p2} and \texttt{p4}.
The promises \texttt{p3} and \texttt{p5}, returned by the \texttt{then} calls to \texttt{p1} and \texttt{p3}, resolve respectively when \texttt{p2} and \texttt{p4} resolve.
This behavior allow to arrange the callback in a flat cascade of calls, instead of an imbrication of calls and callback.

\begin{code}[js, %
             caption={Example of a sequence of promise}, %
             label={lst:promises-sequence}] %
var p1 = my_fn_1(arg)

var p3 = p1.then(function onSuccess(result) {
  // do something with result to get arg
  var p2 = my_fn_2(arg);
  return p2;
}, function onErrors(reason) {
  // do something with the reason of the error ...
})

var p5 = p2.then(function onSuccess(result) {
  // do something with result to get arg
  var p4 = my_fn_3(arg);
  return p4;
}, function onErrors(reason) {
  // do something with the reason of the error ...
})

p5.then(function onSuccess(result) {
  // do something with result ...
}, function onErrors(reason) {
  // do something with the reason of the error ...
});
\end{code}

The convention for callbacks used by Promises differs from the convention used in \textit{Node.js}.
The former uses two callbacks, while the latter uses the only one, with the \textit{error-first} convention.
This difference is mainly syntactic, and is irrelevant to the transformation from imbrication to sequence.
To remove the syntactic difference from the transformation presented in this paper, we present a new specification, Due, in section \ref{section:due}.
Dues are essentially similar to Promises, and use the \textit{error-first} convention, predominant in \textit{Node.js}.

% \subsubsection{Specification}

% At its creation, the promise expects a callback containing the deferred computation.
% This callback is called with two functions as arguments, \texttt{resolve} to fulfill, and \texttt{reject} to reject the promise after the computation.
% % \textbf{$\warning$} The function \texttt{resolve} does \textbf{not} resolve the promise, it fulfills it.
% After its creation, the promise exposes a \texttt{then} method expecting a callback to continue the execution after the deferred computation.

% Any Promise object is in one of three mutually exclusive states: fulfilled, rejected, and pending.

% A promise \texttt{p} is fulfilled when the function \texttt{resolve} is called.
% A call to \texttt{p.then(onFulfill, onReject)} immediately call the function \texttt{onFulfill}.
% A promise \texttt{p} is rejected when the function \texttt{reject} is called.
% A call to \texttt{p.then(onFulfill, onReject)} immediately call the function \texttt{onReject}.
% A promise is pending if it is neither fulfilled nor rejected.
% A promise is settled if it is not pending, \textit{i.e.} if it is either fulfilled or rejected.
% A promise is resolved if it is settled or if it has been locked in to match the state of another promise.
% Attempting to resolve or reject a resolved promise has no effect.
% A promise is unresolved if it is not resolved.
% An unresolved promise is always in the pending state.
% A resolved promise may be pending, fulfilled or rejected.

% The \texttt{Promise} object exposes these methods :
% \begin{description}
% \item[\texttt{Promise.all(iterable)}] Returns a promise that resolves when all of the promises in the iterable argument have resolved.
% \item[\texttt{Promise.prototype.catch(onRejected)}] Appends a rejection handler callback to the promise, and returns a new promise resolving to the return value of the callback if it is called, or to its original fulfillment value if the promise is instead fulfilled.
% \item[\texttt{Promise.prototype.then(onFulfilled, onRejected)}]~\\ Appends fulfillment and rejection handlers to the promise, and returns a new promise resolving to the return value of the called handler. 
% \item[\texttt{Promise.race(iterable)}] Returns a promise that resolves or rejects as soon as one of the promises in the iterable resolves or rejects, with the value or reason from that promise.
% \item[\texttt{Promise.reject(reason)}] Returns a Promise object that is rejected with the given reason.
% \item[\texttt{Promise.resolve(value)}] Returns a Promise object that is resolved with the given value.
% If the value is a \textit{thenable}, \textit{i.e.} has a method \texttt{then}, the returned promise will follow that \textit{thenable}, adopting its eventual state; otherwise the returned promise will be fulfilled with the value.
% \end{description}

% We present in section \ref{section:spimpl} a simple implementation of Promise in Javascript.
% We only implement \texttt{then}, \texttt{resolve} and \texttt{reject} to keep the implementation concise.
%  % as they are the only methods we use for this equivalence.
% The method \texttt{catch} is redundant with the method \texttt{then}.
% The implementation for the methods \texttt{all} and \texttt{race} are out of scope in this paper.
% However, we present equivalences for both in section \ref{section:all-race}.


\section{Dues} \label{section:due}

We present a simpler alternative to promises in Javascript called \textit{Due}.
A due is an object that is used as a placeholder for the eventual results of a deferred computation.
The method \texttt{then} of a due expects only one callback, following the convention \textit{error-first}, like in \textit{Node.js}.%\footnotemark[\ref{ftn:error-first}].
While the method \texttt{then} of a promise expects two callbacks, \texttt{onSuccess} and \texttt{onErrors}. 

Any Due object is in one of two mutually exclusive states: settled or pending.
At its creation, the due expects a callback containing the deferred computation.
This callback is called with the function \texttt{settle} as argument, to settle the due.
After its creation, the due exposes a \texttt{then} method expecting a callback to continue the execution after the deferred computation.
Similarly to Promise, to allow cascading, the method \texttt{then} returns a Due which resolve when the due returned by its callbacks resolve.

\begin{code}[js, %
             caption={Example of a due}, %
             label={lst:then}] %
var due = my_fn(arg)

due.then(function callback(error, result) {
  if (!error) {
    // do something with result ...
  } else {
    // do something with error ...
  }
});
\end{code}

In listing \ref{lst:then}, \texttt{due} is settled when the function \texttt{settle} is called.
If \texttt{due} is settled, a call to \texttt{due.then(onSettlement)} immediately call the function \texttt{onSettlement}.
A due is pending if it is not settled.
A due is resolved if it is settled or if it has been locked in to match the state of another due.
Attempting to settle a resolved due has no effect.
A resolved due may be pending or settled, while an unresolved due is always in the pending state.
The \texttt{Due} object only exposes the \texttt{then} method.
% \textbf{\texttt{Due.prototype.then(onSettlement)}}\\
% Appends settlement handlers to the due, and returns a new due resolving to the return value of the called handler.
% If the value is a \textit{thenable}, \textit{i.e.} has a method \texttt{then}, the returned due will follow that \textit{thenable}, adopting its eventual state; otherwise the returned due will be fulfilled with the value.
We present in section \ref{section:dueimpl} a simple implementation of Due in Javascript.
