\section{Definitions} \label{section:definitions}

\subsection{Callback} \label{section:definitions:continuation}

A callback is a function passed as a parameter to a function call.
It is invoked by the callee to continue the execution with data not available in the caller context.
We distinguish three kinds of callbacks.

\begin{description}
  \item[Iterators] are functions called for each item in a set, often synchronously.
  \item[Listeners] are functions called asynchronously for each event in a stream.
  \item[Continuations] are functions called asynchronously once a result is available.
\end{description}

As we will see later, Promises are designed as placeholders for a unique outcome.
Iterators and Listeners are invoked multiple times resulting in multiple outcomes.
Only continuations are equivalent to Promises.
Therefore, \textbf{we focus on continuations} in this paper.

Callbacks are often mistaken for continuations ; the former are not inherently asynchronous while the latter are.
In a synchronous paradigm, the sequentiality of the execution flow is trivial.
An operation needs to complete before executing the next one.
On the other hand, in an asynchronous paradigm, parallelism is trivial, operations are executed in parallel by default.
The sequentiality of operations needs to be explicit.
Continuations are the functional way of providing this control over \textbf{the sequentiality of the asynchronous execution flow}.

A \textbf{continuation} is a function passed as an argument to allow the callee not to block the caller until its completion.
The caller is able to continue the execution with other operations in parallel.
The continuation is invoked later, at the termination of the callee to continue the execution as soon as possible and process the result ; hence the name continuation.
It provides a necessary control over the asynchronous execution flow.
It also brings a control over the data flow to replace the return statement at the end of a synchronous function.
At its invocation, the continuation retrieves both the the execution flow and the result.

The convention on how to hand back the result must be common for both the callee and the caller.
For example, in \textit{Node.js}, the signature of a continuation uses the \textit{error-first} convention.
% \ftnt{https://docs.nodejitsu.com/articles/errors/what-are-the-error-conventions}
% \ftnt{http://programmers.stackexchange.com/questions/144089/different-callbacks-for-error-or-error-as-first-argument} convention.
The first argument contains an error or \texttt{null} if no error occurred ; then follows the result.
Listing \ref{lst:continuation} is a pattern of such a continuation.
However, continuations don't impose any conventions ; indeed, other conventions are used in the browser.

\begin{code}[js, %
             caption={Example of a continuation}, %
             label={lst:continuation}] %
my_fn(input, function continuation(error, result) {
  if (!error) {
    console.log(result);
  } else {
    throw error;
  }
});
\end{code}

% The continuation allows to continue the execution sequentially, after the result of \textit{my_fn} is available. 
% When continuations are defined inside the call, like \textit{continuation}, the sequence of deferred execution results in an intricate imbrication of calls and continuations, like in listing \ref{lst:cbhell}.
The callback hell occurs when many asynchronous calls are arranged to be executed sequentially.
Each consecutive operation adds an indentation level, because it is nested inside the continuation of the previous operation.
% That is when each caller must wait for the result before calling the next function.
It produces an imbrication of calls and function definitions, like in listing \ref{lst:cbhell}.
We say that continuations lack the \textbf{chained composition} of multiple asynchronous operations.
Promises allow to arrange such a sequence of asynchronous operations in a more concise and readable way.


\begin{code}[js, %
             caption={Example of a sequence of continuations}, %
             label={lst:cbhell}] %
my_fn_1(input, function cont(error, result) {
  if (!error) {
    my_fn_2(result, function cont(error, result) {
      if (!error) {
        my_fn_3(result, function cont(error, result) {
          if (!error) {
            console.log(result);
          } else {
            throw error;
          }
        });
      } else {
        throw error;
      }
    });
  } else {
    throw error;
  }
});
\end{code}

\subsection{Promise} \label{section:definitions:promise}

% TODO insert these :
% Promise also provide few methods to enhance the asynchronous control flow tools\footnote{\texttt{all} and \texttt{race}}.
% There is, in Javascript, numerous Promise implementations\footnote{37 different implementations in Javascript \url{https://github.com/promises-aplus/promises-spec/blob/master/implementations.md}}.

% This section is based on the Promises section of the specification in ECMAScript 6 Harmony\ftnt{https://people.mozilla.org/~jorendorff/es6-draft.html\#sec-promise-objects} and the Promises page on the Mozilla Developer Network\ftnt{https://developer.mozilla.org/en/docs/Web/JavaScript/Reference/Global_Objects/Promise}.

In a synchronous paradigm, the sequentiality of the execution flow is trivial.
While in an asynchronous paradigm, this control is provided by continuations.
Promises provide a unified \textbf{control over the execution and data flow} for both paradigms.
The specification\ftnt{https://people.mozilla.org/~jorendorff/es6-draft.html\#sec-promise-objects} defines a \textbf{Promise} as an object that is used as a placeholder for the eventual outcome of a deferred (and possibly asynchronous) operation.
% A Promise is an object returned by a function to represent its result
Promises expose a \texttt{then} method which expects a continuation to continue with the result ; this result being synchronously or asynchronously available.

% However, unlike continuations, the Promises specification imposes a convention on how to handle the result.
% Because it is possibly unavailable synchronously, it still requires a continuation to defer the execution when the result is made available.
% A promise requires two continuations to defer the execution in case of errors.
% These two continuations are passed to the \texttt{then} method of the promise, like illustrated in listing \ref{lst:then}.

Promises force another control over the execution flow.
According to the outcome of the operation, they call one function to continue the execution with the result, or another to handle errors.
This \textbf{conditional execution} is indivisible from the Promise structure.
As a result, Promises impose a convention on how to hand back the outcome of the deferred computation, while classic continuations leave this conditional execution to the developer.
% As a result of this difference, Promises and continuations use two different conventions to handle errors and results.
% The two conventions are illustrated in listings \ref{lst:continuation} and \ref{lst:then}.

\begin{code}[js, %
             caption={Example of a promise}, %
             label={lst:then}] %
var promise = my_fn_pr(input)

promise.then(function onSuccess(result) {
  console.log(result);
}, function onErrors(error) {
  throw error;
});
\end{code}

% Continuations lack the chained composition of asynchronous operations.
Promises are designed to fill the lack of chained composition from continuations.
They allow to arrange successions of asynchronous operations as a \textbf{chain of continuations}, by opposition to the imbrication of continuations illustrated in listing \ref{lst:cbhell}.
That is to arrange them, one operation after the other, in the same indentation level.
% The \texttt{then} method synchronously returns a Promise linked with the Promise asynchronously returned by its continuation.
% This link allow to compose chains of asynchronous operations.

The listing \ref{lst:promises-sequence} illustrates this chained composition.
The functions \texttt{my_fn_pr_2} and \texttt{my_fn_pr_3} return promises when they are executed, asynchronously.
Because these promises are not available synchronously, the method \texttt{then} synchronously returns intermediary Promises.
The latter resolve only when the former resolve.
This behavior allows to arrange the continuations as a flat chain of calls, instead of an imbrication of continuations.

\begin{code}[js, %
             caption={A chain of Promises is more concise than an imbrication of continuations}, %
             label={lst:promises-sequence}] %
my_fn_pr_1(input)
.then(my_fn_pr_2, onError)
.then(my_fn_pr_3, onError)
.then(console.log, onError);

function onError(error) {
  throw error;
}
\end{code}

The Promises syntax is more concise, and also more readable because it is closer to the familiar synchronous paradigm.
Indeed, Promises allow to arrange both the synchronous and asynchronous execution flow with the same syntax.
It allows to easily arrange the execution flow in parallel or in sequence according to the required causality.
This control over the execution lead to a modification of the control over the data flow.
Programmers are encouraged to arrange the computation as series of coarse-grained steps to carry over inputs.
In this sense, Promises are comparable to the data-flow programming paradigm.

\subsection{From Continuation to Promise} \label{seciton:definitions:analysis}

As detailed in the previous sections, continuations provide the control over \textbf{the sequentiality of the asynchronous execution flow}.
Promises improve this control to allow \textbf{chained compositions}, and unify the syntax for the synchronous and asynchronous paradigm.
This chained composition brings a greater clarity and expressiveness to source codes.
At the light of these insights, it makes sense for a developer to switch from continuations to Promises.
However, the refactoring of existing code bases might be an operation impossible to carry manually within reasonable time.
We want to automatically transform this sequentiality from an imbrication to a chained composition.

To develop further this transformation, we identify two steps.
The first is to provide an equivalence between a continuation and a Promise.
The second is the composition of this equivalence.
Both steps are required to transform imbrications of continuations into chains of Promises.
% to be able to compose this equivalence for imbrications of continuations to obtain chains of Promises.

Because Promises bring composition, the first step might seem trivial as it does not imply any imbrication to compose.
However, as explained in section \ref{section:definitions:promise}, Promises impose a control over the execution flow that continuations leave free.
This control induces a common convention to hand back the outcome to the continuation.

In the Javascript landscape, there is no dominant convention for handing back outcomes to continuations.
In the browser, many conventions coexist.
For example, \textit{jQuery}'s \texttt{ajax}\ftnt{http://api.jquery.com/jquery.ajax/} method expects an object with different continuations for success and errors.
\textit{Q}\ftnt{http://documentup.com/kriskowal/q/}, a popular library to control the asynchronous flow, exposes two methods to define continuations : \texttt{then} for successes, and \texttt{catch} for errors.
Conventions for continuations are very heterogeneous in the browser.
On the other hand, \textit{Node.js} API always used the \textit{error-first} convention, encouraging developers to provide libraries using the same convention.
In this large ecosystem the \textit{error-first} convention is predominant.
All these examples uses different conventions than the Promise specification detailed in section \ref{section:definitions:promise}.
They present strong semantic differences, despite small syntax differences.
% Some conventions include the conditional execution over the outcome, while other conventions let developers provide it.
% These conventions uses different control-flow.

To translate the different conventions into the Promises one, the compiler would need to identify them.
Such an identification might be possible with a static analysis. \comment{todo references}
However, impracticable because of the heterogeneity.
Indeed, in the browser, each library seems to provide its own convention.

In this paper, we are interested in the transformation from imbrications to chains, not from one convention to another.
The \textit{error-first} convention, used in \textit{Node.js}, is likely to represent a large, coherent code base to test the equivalence over.
For this reason, \textbf{we focus only on the \textit{error-first} convention}.
So, The compiler is only able to compile code that follows this convention.
The convention used by Promises is incompatible.
We propose an alternative specification to Promise following the \textit{error-first} convention.
In the next section we present this specification, called Due.

The choice to focus on \textit{Node.js} is also motivated by our intention to later compare the chained sequentiality of Promises with the data-flow paradigm.
\textit{Node.js} is a framework for real-time web applications ; it allows to manipulate flow of I/O data.

% In section \ref{section:equivalence}, we explain the two steps of the transformation from continuations to Dues.

\subsection{Dues} \label{section:definitions:due}

In this section, we present \textit{Dues}, a simplification of the Promise specification.
A Due is an object used as placeholder for the eventual outcome of a deferred operation.
Dues are essentially similar to Promises, except for the convention to hand back outcomes.
% They leave the control over the conditional execution over the outcome to the developer.
They use the \textit{error-first} convention, like \textit{Node.js}, as illustrated in listing \ref{lst:due}.
The implementation of Dues and its tests are in appendix \ref{section:dueimpl}.
A more in-depth description of Dues and their creation follows in the next paragraph.
% The \texttt{mock} method is implemented in listing \ref{lst:due-creation}.
% While a promise expects two continuations, \texttt{onSuccess} and \texttt{onErrors}, the method \texttt{then} of a due expects only one continuation, following the convention \textit{error-first}.
% \footnotemark{\ref{ftn:error-conventions}}
% \footnotemark{\ref{ftn:error-first}}.

\begin{code}[js, %
             caption={Example of a due}, %
             label={lst:due}] %
var my_fn_due = require('due').mock(my_fn);

var due = my_fn_due(input);

due.then(function continuation(error, result) {
  if (!error) {
    console.log(result);
  } else {
    throw error;
  }
});
\end{code}

A due is typically created inside the function which returns it.
In listing \ref{lst:due}, the \texttt{mock} method create a Due-compatible function from \texttt{my\_fn}.
We illustrate the creation of a Due with the \texttt{mock} method, in listing \ref{lst:due-creation}.

At its creation, line \ref{lst:due-creation:new}, the due expects a callback containing the deferred operation, \texttt{my\_fn}.
This callback is executed synchronously with the function \texttt{settle} as argument to settle the Due, synchronously or asynchronously.
% Indeed, the operation might be synchronous, or asynchronous.
The callback invokes the deferred operation line \ref{lst:due-creation:call}.
\texttt{my\_fn} being asynchronous, it expects a callback as last argument : \texttt{settle}.
At the end of the operation, the deferred operation calls \texttt{settle} to settle the Due and save the outcome.
% A Due is in one of two mutually exclusive states: settled or pending.
Settled or not, the created Due is synchronously returned.
Its \texttt{then} method allows to define a continuation to retrieve the saved outcome, and continue the execution after its settlement.
If the deferred operation is synchronous, the Due settles during its creation and the \texttt{then} method immediately calls this continuation.
If the deferred operation is asynchronous, this continuation is called during the Due settlement.

% This continuation is defined by the \texttt{then} method.
% After the settlement of the Due, its continuation is executed with the outcome.
% Dues expose a \texttt{then} method expecting a continuation to continue the execution after its settlement.

\begin{code}[js, %
             caption={Creation of a due}, %
             label={lst:due-creation}] %
Due.mock = function(my_fn) {
  return function() {
    var _args = Array.prototype.slice.call(arguments),
        _this = this;

    return new Due(function(settle) {  //@\label{lst:due-creation:new}@
      _args.push(settle);
      my_fn.apply(_this, _args); //@\label{lst:due-creation:call}@
    })
  }
}
\end{code}

The composition of Dues is essentially the same than for Promises, detailed in section \ref{section:definitions:promise}.
% This composition is explained in details in section \ref{section:definitions:promise}. %, and illustrated specifically for Dues in listing \ref{lst:dues-sequence}.
Through this chained composition, Dues arrange the execution flow as a sequence of actions to carry on inputs.

% \begin{code}[js, %
%              caption={Dues are chained like Promises}, %
%              label={lst:dues-sequence}] %
% var Due = require('due');

% var my_fn_due_1 = Due.mock(my_fn_1),
%     my_fn_due_2 = Due.mock(my_fn_2),
%     my_fn_due_3 = Due.mock(my_fn_3);

% my_fn_1(input)
% .then(my_fn_2)
% .then(my_fn_3)
% .then(console.log);
% \end{code}

This simplified specification adopts the same convention than \textit{Node.js} continuations to hand back outcomes.
Therefore, the equivalence between a continuation and a Due is trivial.
% This equivalence, and its composition are explained in details in section \ref{section:equivalence}.
Dues are admittedly tailored for this paper, hence, they are not designed to be written by developers, like Promises are.
They are an intermediary steps between classical continuation and Promises.
We highlight in the section \ref{section:equivalence} the equivalence between the two latter.
But our goal is also to highlight the similitudes between the chained composition, and data-flow paradigms.
Indeed, both arrange the computation as series of chained operations to carry, in parallel and in sequence.


% In listing \ref{lst:due}, \texttt{due} is settled when the function \texttt{settle} is called.
% If \texttt{due} is settled, a call to \texttt{due.then(onSettlement)} immediately call the function \texttt{onSettlement}.
% A due is pending if it is not settled.
% A due is resolved if it is settled or if it has been linked with another due.
% Attempting to settle a resolved due has no effect.
% A resolved due may be pending or settled, while an unresolved due is always in the pending state.
% The \texttt{Due} object only exposes the \texttt{then} method.
% \textbf{\texttt{Due.prototype.then(onSettlement)}}\\
% Appends settlement handlers to the due, and returns a new due resolving to the return value of the called handler.
% If the value is a \textit{thenable}, \textit{i.e.} has a method \texttt{then}, the returned due will follow that \textit{thenable}, adopting its eventual state; otherwise the returned due will be fulfilled with the value.
% We present in appendix \ref{section:dueimpl} a simple implementation of Due in Javascript.
