\section*{Abstract}

Javascript is the prevalent scripting language for the web.
It lets web pages register callbacks to react to user events.
A callback is a function to be invoked later with a result currently unavailable.
This pattern also proved to respond efficiently to remote requests.
Javascript is currently used to implement complete web applications.
However, callbacks are ill-suited to arrange a large asynchronous execution flow.
\textit{Promises} are a more adapted alternative.
They provide a unified control over both the synchronous and asynchronous execution flows.

The next version of Javascript proposes to replace callbacks with Promises.
% Promises are about to replace callbacks.
This paper brings the first step toward a compiler to help developers prepare this shift.
We present an equivalence between callbacks and Dues.
The latter are a simpler specification of Promises developed for the purpose of this demonstration.
From this equivalence, we implement a compiler to transform an imbrication of callbacks into a chain of Dues.
This equivalence is limited to \textit{Node.js}-style asynchronous callbacks declared \textit{in situ}.
We evaluate our compiler over 64 \textit{npm} packages, 9 of them present compatible callbacks and compile successfully.

We consider this shift to be a first step toward the merge of concepts from the data-flow programming model into the imperative programming model.

% \vfill\eject