
% \section{Promises} \label{section:promise}

% This section is based on the Promises section of the specification in ECMAScript 6 Harmony\footnote{\url{https://people.mozilla.org/~jorendorff/es6-draft.html\#sec-promise-objects}} and the Promises page on the Mozilla Developer Network\footnote{\url{https://developer.mozilla.org/en/docs/Web/JavaScript/Reference/Global_Objects/Promise}}.

% The specification defines a promise as \textit{an object that is used as a placeholder for the eventual results of a deferred (and possibly asynchronous) computation.
% Any Promise object is in one of three mutually exclusive states: fulfilled, rejected, and pending.}

% At its creation, the promise expects a callback containing the deferred computation.
% This callback is called with two functions as arguments, \texttt{resolve} to fulfill, and \texttt{reject} to reject the promise after the computation.
% % \textbf{$\warning$} The function \texttt{resolve} does \textbf{not} resolve the promise, it fulfills it.
% After its creation, the promise exposes a \texttt{then} method expecting a callback to continue the execution after the deferred computation.

% A promise \texttt{p} is fulfilled when the function \texttt{resolve} is called.
% A call to \texttt{p.then(onFulfill, onReject)} immediately call the function \texttt{onFulfill}.
% A promise \texttt{p} is rejected when the function \texttt{reject} is called.
% A call to \texttt{p.then(onFulfill, onReject)} immediately call the function \texttt{onReject}.
% A promise is pending if it is neither fulfilled nor rejected.
% A promise is settled if it is not pending, \textit{i.e.} if it is either fulfilled or rejected.
% A promise is resolved if it is settled or if it has been locked in to match the state of another promise.
% Attempting to resolve or reject a resolved promise has no effect.
% A promise is unresolved if it is not resolved.
% An unresolved promise is always in the pending state.
% A resolved promise may be pending, fulfilled or rejected.

% The \texttt{Promise} object exposes these methods :
% \begin{description}
% \item[\texttt{Promise.all(iterable)}] Returns a promise that resolves when all of the promises in the iterable argument have resolved.
% \item[\texttt{Promise.prototype.catch(onRejected)}] Appends a rejection handler callback to the promise, and returns a new promise resolving to the return value of the callback if it is called, or to its original fulfillment value if the promise is instead fulfilled.
% \item[\texttt{Promise.prototype.then(onFulfilled, onRejected)}]~\\ Appends fulfillment and rejection handlers to the promise, and returns a new promise resolving to the return value of the called handler. 
% \item[\texttt{Promise.race(iterable)}] Returns a promise that resolves or rejects as soon as one of the promises in the iterable resolves or rejects, with the value or reason from that promise.
% \item[\texttt{Promise.reject(reason)}] Returns a Promise object that is rejected with the given reason.
% \item[\texttt{Promise.resolve(value)}] Returns a Promise object that is resolved with the given value.
% If the value is a \textit{thenable}, \textit{i.e.} has a method \texttt{then}, the returned promise will follow that \textit{thenable}, adopting its eventual state; otherwise the returned promise will be fulfilled with the value.
% \end{description}

% We present in section \ref{section:spimpl} a simple implementation of Promise in Javascript.
% We only implement \texttt{then}, \texttt{resolve} and \texttt{reject} to keep the implementation concise.
%  % as they are the only methods we use for this equivalence.
% The method \texttt{catch} is redundant with the method \texttt{then}.
% The implementation for the methods \texttt{all} and \texttt{race} are out of scope in this paper.
% However, we present equivalences for both in section \ref{section:all-race}.
