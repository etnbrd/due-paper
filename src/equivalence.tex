\section{Equivalence} \label{section:equivalence}

% The previous section defined continuations, the \textit{error-first} convention, and Dues.
% , a simpler Promises specification bringing chained composition.
We present the semantic equivalence between a continuation and a Due.
This equivalence allows to transform imbrications of continuations into chains of Dues.
% , and then the composition of this equivalence to develop further the transformation from imbrications into chains.
% While developing the equivalence, we chose to focus only on a subset of the Javascript programs.
% As we will see, it exposes some limitations.
% The callee must be asynchronous.
% The continuation must be a \textit{FunctionExpression}
% The continuation and the callee must not return a significant value.

% In the listings of the previous section, \texttt{my_fn} either expects a continuation, or returns a Due.



% The main advantage for developers to use Dues, is to flatten the overlapping continuations into a more readable, linear sequence.
% The nesting of continuations only occurs when they are defined by \textit{FunctionExpressions}\footnote{\url{http://www.ecma-international.org/ecma-262/5.1/\#sec-11.2.5}}.
% When the continuation is not declared \textit{in situ}, it avoids the imbrication of function declarations and calls.
% We focus only on the modification of continuation declared \textit{in situ}.
% Moreover, the transformation is \textit{sound} only when manipulating \textit{FunctionExpressions}, as explained in section \ref{section:equivaences:general}.
% This equivalence would not improve readability.
% Moreover, it would require heavier manipulation of the source code, as explained in section \ref{section:equivaences:general}.

% The transformations presented modifies the syntax of the asynchronous call.
% The asynchronous function needs to be modified to return a Due, instead of expecting a continuation.
% For the demonstrations, we use the function \texttt{my_fn} in listing \ref{lst:my-fn}.
% It both expects a callback and returns a Due.
% There is no libraries compatible with both callback and Dues, like \texttt{my_fn}.
% However, the Due library provide a function \texttt{mock} to transform a function expecting continuation into a function returning a Due.
% We don't focus neither on the replacement of these libraries, nor on the detection of their methods in the source code.
% We expect the continuations to be already screened out from other callbacks, either by a developer, or by another automated tool.
% We address this problem in section \ref{section:compiler:lib-compilation}.

To illustrate the transformations, we use the function \texttt{my_fn} in listing \ref{lst:my-fn}.
This function is tailored for this transformation, it both expects a callback and returns a Due.
The transformation modifies the required signature of the callee.
In the source of the transformation, the callee expects a continuation, while in the result of the transformation, the callee returns a Due.
The modification of the signature of the callee is addressed in section \ref{section:limits:mock}.

\includecode{js, %
             caption={\texttt{my_fn} expects a callback, and returns a Due}, %
             label={lst:my-fn}}{snippets/my-fn.js}


\subsection{Simple equivalence} \label{section:equivalence:simple}

% As explained in section \ref{section:definitions:callback}, a continuation is a function passed as argument to defer its execution, like in listing \ref{lst:ct-ex}.
% As explained in section \ref{section:due}, a Due is an object to defer a computation, and exposes a method \texttt{then} to continue the execution after the deferred computation, like in listing \ref{lst:du-ex}.
% Because the difference between continuations and dues is the composition, the difference between the listings \ref{lst:ct-ex} and \ref{lst:du-ex} is mainly syntactical.

Continuations provide a control over the sequentiality of the asynchronous execution flow, and Dues brings the chained composition of this control.
The equivalence between a single continuation and a Due does not involve composition, so the manipulation to transform listing \ref{lst:ct-ex} into \ref{lst:du-ex} is trivial.
It consist of calling the method \texttt{then} of the returned Due, and moving \texttt{continuation} to the arguments of this new call.

\includecode{js, %
             caption={A simple continuation}, %
             label={lst:ct-ex}
             }
             {snippets/ct-ex.js}

\includecode{js, %
             caption={The syntax of a continuation and its Due equivalence are very similar}, %
             label={lst:du-ex}
             }
             {snippets/du-ex.js}

% Javascript defines functions as first-class citizens.
% It allows the definition of \textit{Functions Expressions} to be passed as arguments, like \texttt{continuation}.
% Moreover, Javascript defines scopes at the function level.
The relocation of the evaluation of \texttt{continuation} might alter the execution order, hence the semantic.
But the definition of a \textit{Function Expression} is free of side-effects.
Indeed, it leaves intact the scope in which it is defined.
% Indeed, the identifier \texttt{continuation} is available only inside itself, for recursion.
The manipulation conserves the semantic, it is \textit{sound}.

To summarize the specification, these \textit{Expressions} possibly return a function.
\begin{itemize}
  \item a \textit{FunctionExpression},
  \item an \textit{ArrowFunction},
  \item an \textit{Identifier}, a \textit{MemberExpression} or a \textit{ThisExpression} pointing to a function,
  \item a \textit{CallExpression}, \textit{YieldExpression} returning a function, and
  \item a \textit{NewExpression} creating a new function,
\end{itemize}

We focus only on continuations expressed as \textit{Function Expressions}.
For other types of continuations, the manipulation is either irrelevant, or unsound.
Respectively, either the continuation prevent nesting, or its evaluation causes side-effects.
For example, when using \textit{Identifiers}, it is impossible to nest continuations, the equivalence would be irrelevant.
When using \textit{Immediately-Invoked Function Expression} (IIFE), a common pattern to return a closure, the manipulation is unsound.
Before the manipulation, the evaluation of this expression would occur before the call to \texttt{my_fn}.
While, after the manipulation, it would occur after the call to \texttt{my_fn}.
The manipulation would prevent side-effects from happening before the call to \texttt{my_fn}.
The manipulation is \textit{sound} and relevant only when manipulating \textit{FunctionExpression}.

\subsection{Composition of nested continuations} \label{section:equivalence:composition}

The equivalence previously presented is incomplete, it leaves sequential operations nested.
To transform an imbrication of continuations into a chain of Dues, we need to assure the composition of this simple equivalence.
An example of nested continuations is illustrated in listing \ref{lst:ct-seq}.
The semantic equivalent chain of Dues is illustrated in listing \ref{lst:du-seq}.
The composition of this equivalence requires at least two additional transformations.

\begin{itemize}
  \item The nested continuation \texttt{cont2} is chained in the same indentation level as the first one, by a second call to the \texttt{then} method, line \ref{lst:du-seq:ctdef2}.
  This second call refers to the intermediary Due returned by the first call to the \texttt{then} method.
  \item For this chain to be possible, this intermediary Due must be linked with the Due returned by the nested call to \texttt{my_fn}.  
  % which the second call refers to.
  The Due from the nested asynchronous function \texttt{cont1} is retrieved, line \ref{lst:du-seq:ctdef2}, and returned line \ref{lst:du-seq:ret} to be internally linked to the intermediary Due.
  \item The definition of \texttt{shared_identifier} is relocated in the global scope to be accessible by both \texttt{cont1} and \texttt{cont2}. The accessibility of identifiers is developed in section \ref{section:limits:disjunction}.
\end{itemize}

\includecode{js, %
             caption={Overlapping continuations definitions}, %
             label={lst:ct-seq}
             }
             {snippets/ct-seq.js}

\includecode{js, %
             caption={Sequential continuations definitions using Dues}, %
             label={lst:du-seq}
             }
             {snippets/du-seq.js}

The composition of the equivalence leads to some semantical differences.
It is unsound in some corner cases.
In the next section, we explore the limits of the equivalence to ensure soundness, and allow automation.
