\section{Equivalence}

We present two equivalences to transform callbacks into promises in section \ref{section:general}, and \ref{section:sequence}.
These equivalences are \textit{sound} because it encompass every possible case of argument for a callback.
These equivalences are not easily readable for the same reason.
A compiler would distinguish many different cases and adapt for each one a more precise, and readable, equivalence.
We present section \ref{section:simple} such a more precise equivalence.
In section \ref{section:more}, We present some hints to bring these equivalences further near human comprehension.

The source must use libraries compatible with promises.
So the functions using callback before compilation must returns a promise after compilation.
\texttt{my_fn} in listing \ref{lst:my-fn} is a function both expecting a callback and returning a promise.
There is no known libraries compatible with both callback and promises, like \texttt{my_fn}.
However, there exist some libraries to trivially replace the use of callback for closures, like \textit{q-io}\footnote{\url{https://www.npmjs.org/package/q-io}}.
We don't focus neither on the detection and replacement of these libraries, nor on the detection of their methods.

\includecode{js, %
             caption={Example of a function expecting a callback}, %
             label={lst:my-fn}}{snippets/my-fn.js}

\subsection{General equivalence} \label{section:general}

The call to \texttt{my_fn} is a \textit{CallExpression}\footnote{\url{https://people.mozilla.org/~jorendorff/es6-draft.html\#sec-expression-rules}}.
Only be AssignementExpression can be argument of a CallExpression.
Among the AssignementExpressions, all the expressions that possibly returns a callable object, \textit{i.e.} a function, or a method, are :
\begin{itemize}
\item Identifier
\item FunctionExpression
\item ArrowFunction
\item YieldExpression
\item CallExpression
\item MemberExpression
\item this
\end{itemize}
In the listings \ref{lst:cb-simple}, \ref{lst:pr-simple}, \ref{lst:cb-seq} and \ref{lst:pr-seq}, the identifier \texttt{callback} can be replaced with any of these expressions.

During the transformation of the call from the listing \ref{lst:cb-simple} into the one in the listing \ref{lst:pr-simple}, there is several points to notice.
In the listing \ref{lst:cb-simple}, the evaluation of \texttt{callback} is done after the evaluation of \texttt{arg}, but before the evaluation of \texttt{my_fn}.
It is important to keep this evaluation order in the result to keep the semantic in every cases.
We introduce a variable \texttt{__cb} to hold the original callback, and the function \texttt{__cb_def} to define \texttt{__cb} and evaluate \texttt{callback} in the original order.
The function \texttt{__cb_def} returns \texttt{undefined} to keep the array of arguments passed to \texttt{my_fn} intact.
After the call to \texttt{my_fn}, we append a call to \texttt{then} referring to the promise returned by \texttt{my_fn}.
We populate its arguments with two functions, lines \ref{lst:pr-simple:onFulfill} and \ref{lst:pr-simple:onReject}, to call the callback in case of fulfillment, or rejection of the promise, lines \ref{lst:pr-simple:fulfill} and \ref{lst:pr-simple:reject}.
The first function appends \texttt{null} to the arguments to fill the \texttt{error} parameter, as expected bu the callback.
Both functions call the callback with the passed arguments, and returns the value returned by the callback.
This returned value might be a promise.
% We move the execution of \texttt{callback} from the arguments of \texttt{my_fn} to the arguments of \texttt{then}, lines \ref{lst:pr-simple:fulfill} and \ref{lst:pr-simple:reject}.

\includecode{js, %
             caption={A simple callback}, %
             label={lst:cb-simple}
             }
             {snippets/cb-simple.js}

\includecode{js, %
             caption={Promise equivalence of a simple callback}, %
             label={lst:pr-simple}
             }
             {snippets/pr-simple.js}

\subsection{Simple equivalence} \label{section:simple}

The two listings \ref{lst:cb-simple} and \ref{lst:pr-simple} are semantically equivalent in all cases.
But as said earlier, the result looses readability.
We present in listing \ref{lst:cb-exp} and \ref{lst:pr-exp} a more precise equivalence where the callback is a FunctionExpression.
This more precise equivalence is slightly more readable than the more general one.
However it keeps callback hell intact, and introduce code duplication.
We address the callback hell in section \ref{section:sequence}, and code duplication in section \ref{section:duplication}.

\includecode{js, %
             caption={A FunctionExpression callback}, %
             label={lst:cb-exp}
             }
             {snippets/cb-exp.js}

\includecode{js, %
             caption={Promise equivalence of a FunctionExpression callback}, %
             label={lst:pr-somple}
             }
             {snippets/pr-exp.js}


\subsection{Sequence equivalence} \label{section:sequence}

To transform nested callbacks into sequential promises, we present the promise equivalence of two callback nested one into the other.
The nested callback is translated into a second call to \texttt{then}, referring to the promise returned by the first call to \texttt{then}.
If this callback is encapsulating another callback, this third callback would be translated into a third call to \texttt{then}, referring to the promise returned by the second call to \texttt{then}.
And so on.
If we can flatten two nested callbacks, then we can flatten \textit{n} nested callbacks.

Listing \ref{lst:cb-seq} present two callback, \texttt{cb1} line \ref{lst:cb-seq:cb1} and \texttt{cb2} line \ref{lst:cb-seq:cb2}, nested into one another.
We turned the variable \texttt{__cb} into an array to hold both callbacks.
The callbacks are pushed after evaluation during the call to \texttt{__cb_def}, lines \ref{lst:pr-seq:cbdef1} and \ref{lst:pr-seq:cbdef2}.
Before executing the callbacks, they are extracted from \texttt{__cb}.
This transformation into array is \textit{sound} because
\begin{itemize}
\item the functions \texttt{onFulFill1} and \texttt{onReject1} from the first \texttt{then} are assured to execute before the functions \texttt{onFulFill2} and \texttt{onReject2} from the second \texttt{then}, so the callbacks are extracted in the correct order ;
\item only one of the functions \texttt{onFulFill} and \texttt{onReject} is executed, so the callback are extracted once, and only once ; and
\item those functions are defined in the same scope as \texttt{__cb}, a nested scope using the same variable name would hide the parent one, so it is impossible to conflict with another \texttt{__cb} from another similar transformation.
\end{itemize}
Line \ref{lst:pr-seq:retpr}, we make \texttt{cb1} returns the promise returned by the second call to \texttt{my_fn} line \ref{lst:pr-seq:cbdef2}.
Line \ref{lst:pr-seq:fulfill1} or \ref{lst:pr-seq:reject1}, the functions \texttt{onFulfill1} or \texttt{onReject1} retrieves the promise returned by \texttt{cb1} and return it immediately.
If more than two callback are nested, the variable \texttt{__promise} doesn't need to be turned into an array.
Its definition, line \ref{lst:pr-seq:cbdef1}, and usage, line \ref{lst:pr-seq:retpr}, always happens synchronously.
It could be defined directly inside \texttt{cb1}.

\includecode{js, %
             caption={A nested callback}, %
             label={lst:cb-seq}
             }
             {snippets/cb-seq.js}

\includecode{js, %
             caption={Promise equivalence of a nested callback}, %
             label={lst:pr-seq}
             }
             {snippets/pr-seq.js}

The two listings \ref{lst:cb-seq} and \ref{lst:pr-seq} are not semantically equivalent in all cases.
\begin{itemize}
\item This equivalence is no longer \textit{sound} if the callback is not called asynchronously.
Indeed, if \texttt{m_fn} is synchronous, its second call synchronously calls \texttt{cb2}.
So, the callback in listing \ref{lst:cb-seq} executes between the execution of the two comments \circled{1} and \circled{2}.
While the promise in listing \ref{lst:pr-seq} executes its callback after the two comments \circled{1} and \circled{2}.
Synchronous functions must be detected and avoided to keep the soundness.
\item This equivalence is no longer \textit{sound} if there is more than one callback nested inside another callback : sibling callbacks.
It is impossible to return two promises, one for each callback.
Say $A$ uses the callback $cbA$, and $B$ uses the callback $cbB$ ; $cbA$ and $cbB$ are siblings.
It is different than $A$ and $B$, when both finished, calling both $cbA$ and $cbB$.
This second case is the use case of the \texttt{all} method, addressed in section \ref{section:all-race}.
Using this equivalence with two sibling callbacks breaks the semantic.
To keep the soundness of this equivalence, sibling callbacks must remain siblings.
\end{itemize}
In this equivalence, the structure of callback calls is flattened, but the definition of callbacks are still overlapping.
The callback hell remains, or we introduce code duplication, as explained in section \ref{section:simple}.
The next section address the flattening of the callback definitions.

\subsection{More equivalences} \label{section:more}

As previously announced, these equivalences leave the callback hell intact, because they encompass all possible cases of callback.
More precise equivalences are needed to both flatten the callback hell and avoid code duplication.

\subsubsection{Callback to onFulfill and onReject} \label{section:duplication}

The callback handles both \texttt{error} and \texttt{result}, while the \texttt{then} method expect one function for each.
The compiler analyze and detect how the callback handles them to split it into the two functions to pass to the \texttt{then} method.
The simplest case is a callback not handling errors at all.
Like the conditional branch in listing \ref{lst:callback}.
However, ignoring errors is a bad practice.
Other examples are a conditional branch, like in listing \ref{lst:err-cond}, or an early return, like in listing \ref{lst:err-return}.

\begin{code}[js, %
             caption={A conditional switch to handle errors}, %
             label={lst:err-cond}] %
my_fn('<arg>', function callback(error, result) {
  if (error) {
    // handle the error ...
  } else {
    // do something with result ...
  }
})
\end{code}

\begin{code}[js, %
             caption={An early return to handle errors}, %
             label={lst:err-return}] %
my_fn('<arg>', function callback(error, result) {
  if (error) {
    // handle the error ...

    return
  }
  
  // do something with result ...
})
\end{code}

\subsubsection{all and race} \label{section:all-race}

\paragraph{all}

A callback might needs the completion of multiple operations to proceed.
For example, to count the number of lines in a bunch of files, all files needs to be read.
A simple way to do this is to use the same closure to count all the completed operations.
The actual callback is called only when all operations completes.
An example of this pattern is in listing \ref{lst:all}.
It uses a variable \texttt{count}, line \ref{lst:all:after}, and increment it at each operation completion.
It executes the callback only when the \textit{n} call to \texttt{my_fn} complete, line \ref{lst:all:callback}.
This pattern is equivalent to the promise method \texttt{all}.

\includecode{js, %
             caption={Waiting for the completion of all operation before calling the callback}, %
             label={lst:all}
             }
             {snippets/all.js}

\paragraph{race}

Sometimes, a callback needs the completion of at least one operation to proceed.
For example, to display early result in a broad search.
This is similar to the previous pattern.
Instead of waiting for every operation completion, the closure executes the callback when first called.
To be equivalent to the promise method \texttt{race}, this closure must block all future execution of callback.

\includecode{js, %
             caption={The first completed operation triggers the callback}, %
             label={lst:race}
             }
             {snippets/race.js}

% The complexity of such analyses would require probably more than syntactic equivalences to encompass all cases.