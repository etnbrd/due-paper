\section{Compiler} \label{section:compiler}

We build a compiler to automate the application of this equivalence on existing Javascript repositories.
The compilation process contains two important steps, the identification of the continuations, and the generation of chains.

\subsection{Continuation identification}

The equivalence is applicable only on continuations.
The first compilation step is to identify the continuations among the callbacks.
A continuation is a callback invoked only once, asynchronously.
Spotting a continuation implies to identify these two conditions.
There is no syntactical difference between a synchronous and an asynchronous callee.
And it is impossible to assure a callback to be invoked only once, because the implementation of the callee is often unavailable statically.
Therefore, the identification of continuations is necessarily based on semantical differences.
For this purpose, the compiler would need to have a deep understanding of the control and data flows of the program.
Because of the highly dynamic nature of Javascript, this understanding is either unsound, limited, or complex.
Instead, we choose to leave to the developer the identification of compatible continuations among the identified callbacks.
They are expected to understand the limitations of this compiler, and the semantic of the code to compile.

We provide a simple interface for developers to interact with the compiler.
We built the compiler as a web page.
The compiler is available online\ftnt{compiler-due.apps.zone52.org} to reproduce the tests.

This interaction prevent the complete automation of the individual compilation process.
However, we are working on an automation at a global scale.
We expect to be able to identify of a continuation only based on its callee name, \textit{e.g.} \texttt{fs.readFile}.
We built a service to gather these names along with their identification.
The compiler query this service to present an estimated identification, and send back the final choices to refine it.
In future works, we would like to study the possibility of easing the compilation interaction.

\subsection{Chains}

The compositions of continuations and Dues are arranged differently.
Continuations structure the execution flow as a tree, while the chain of Dues arrange it sequentially.
A parent continuation can have several children, not a Due.
The second compilation step is to identify these imbrications of continuations to transform them into chains.

% We consider an imbrication of continuations as a subtree of the linear execution tree.
The compiler trims each tree of continuations to get chains to translate into Dues.
If a continuation has more than one child, the compiler try to find a legitimate child to continue the chain.
The legitimate child is the only parent among its siblings.
If there is several parents among the children, then none are the legitimate child.
The non legitimate children start a new tree.
This steps yield chains of continuations assured to be transformable into sequences of Dues.
The code generation from these chains is straightforward.






% \comment{TODO insert this}
% We developed the compiler core in node.js Javascript.
% There already exist sets of tools for manipulating code in Javascript.
% We used the Esprima suite of tools.

