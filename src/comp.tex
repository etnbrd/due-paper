\section{Compiler} \label{section:compiler}

We build a compiler to automate the application of this equivalence on existing Javascript projects.
The compilation process contains two important steps, the identification of the continuations, and the generation of chains.

\subsection{Identification of continuations}

The first compilation step is to identify the continuations and their imbrications.
The nested imbrication of callbacks only occurs when they are defined \textit{in situ}.
The compiler detects a function definition within the arguments of a function call.
This detection is based on the syntax, and is trivial.

% The callback hell occurs only when using callbacks defined \textit{in situ}.
% The compiler detects only such callbacks.
% This detection is syntactic.

Not all detected callbacks are continuations, but the equivalence is applicable only on the latter.
A continuation is a callback invoked only once, asynchronously.
Spotting a continuation implies to identify these two conditions.
There is no syntactical difference between a synchronous and an asynchronous callee.
And it is impossible to assure a callback to be invoked only once, because the implementation of the callee is often statically unavailable.
Therefore, the identification of continuations is necessarily based on semantical differences.
To recognize these differences, the compiler would need to have a deep understanding of the control and data flows of the program.
Because of the highly dynamic nature of Javascript, this understanding is either unsound, limited, or complex.
Instead, we choose to leave to the developer the identification of compatible continuations among the identified callbacks.
They are expected to understand the limitations of this compiler, and the semantic of the code to compile.

We provide a simple interface for developers to interact with the compiler.
We built this interface around the compiler in a web page available online\ftnt{compiler-due.apps.zone52.org} to reproduce the tests.
The web technologies allow to quickly build an interface for a wide variety of computing devices.

This interaction prevents the complete automation of the individual compilation process.
However, we are working on an automation at a global scale.
We expect to be able to identify a continuation only based on the name of its callee, \textit{e.g.} \texttt{fs.readFile}.
We built a service to gather these names along with their identification.
The compiler queries this service to present to the developer an estimated identification.
After the compilation, it sends back the identification corrected by the developer to refine the future estimations.
In future works, we would like to study the possibility for such a service to assist, and ease the compilation process.

\subsection{Generation of chains}

The compositions of continuations and Dues are arranged differently.
Continuations structure the execution flow as a tree, while a chain of Dues imposes to arrange it sequentially.
A parent continuation can execute several children, while a Due allow to chain only one.
The second compilation step is to identify the imbrications of continuations, and trim the extra branches to transform them into chains.
% We consider an imbrication of continuations as a subtree of the linear execution tree.
% The compiler trims each tree of continuations to get chains that translate into Dues.

If a continuation has more than one child, the compiler tries to find a single legitimate child to form the longest chain possible.
This legitimate child is the only parent among its siblings.
If there are several parents among the children, none are the legitimate child.
The non legitimate children start a new tree.
This step transform each tree of continuations into several chains of continuations that translate into sequences of Dues.
The code generation from these chains is straightforward from the equivalence.






% \comment{TODO insert this}
% We developed the compiler core in node.js Javascript.
% There already exist sets of tools for manipulating code in Javascript.
% We used the Esprima suite of tools.

