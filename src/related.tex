\section{Related works} \label{section:related}

To our knowledge, our work is the first to explore the transformation from continuations to Promises in Javascript, and to state the similarity between Promises and data-flow programming.
This section presents the various works related to ours.
Our work is based on the previous work on Promises and Futures~\cite{Liskov1988}, and their specifications in Javascript\footnote{\url{https://promisesaplus.com/}}\footnote{\url{https://people.mozilla.org/~jorendorff/es6-draft.html\#sec-promise-objects}}.


Because of its dominant position in the web, Javascript is recently subject to a growing interest in the field of static analysis.
We identify two teams working on static analysis for Javascript.
In the Department of Computing, Imperial College London, S. Maffeis, P. Gardner and G. Smith realised a large body of work around the static analysis of Javascript.
Their work is based around an operational semantic~\cite{Maffeis2008} to bring program understanding~\cite{Smith2011,Gardner2012,Gardner2013,Bodin2014}.
Their goal seems to revolve around security applications of this analysis~\cite{Maffeis2009,Maffeis2009a}.
In a collaboration between the programming language research groups at Aarhus University and Universität Freiburg, P. Thiemann, S. Jensen and A. Møller are working on the static analysis of Javascript.
They presented a tool providing type inference using abstract interpretation~\cite{Thiemann2005,Jensen2009,Jensen2012}.
Their goal is to improve the tools available for Javascript developers~\cite{Andreasen}.
Another example of interest for Javascript static analysis is the adaptation of the points-to analysis from L. Andersen's thesis~\cite{Andersen1994} to Javascript, by D. Jang \textit{et al.}~\cite{Jang2009} and S. Wei \textit{et al.}~\cite{Wei2014}.

The industry seems to follow the same trends.
There are some security tools based on static analysis.
We can cite for example, the company Shape Security\footnote{\url{https://shapesecurity.com/}}.
They developed \textit{Esprima}, a Javascript parser, and a serie of tools to help static analysis.
Facebook released flow\footnote{\url{http://flowtype.org/}} on 26 October 2014, a static type checker for Javascript.

Promises combine controls over the execution and the data flow.
It arrange the execution parts sequentialy and assign the result of one into the inputs of the next.
This arrangement seems similar to some works on the field of functional and data-flow programming~\cite{Johnston2004,Cohen2012,Morrison1994,Kahn1974}.
We consider it a first step in the merge of elements from the data-flow paradigm into the imperative paradigm.
The Functional Reactive Programming paradigm pushes the intrication of data and control-flow even further~\cite{Winograd-Cort2013}.

% Our compiler aggregates user preferences to transparently improve the service for every user.
% To our knowledge, we are the first to use principle for software compilation.
% A good example of similair work is Aviate\footnote{\url{http://aviate.yahoo.com/}}, an android homescreen which automatically organize smartphone applications into existing categories.
% The use of crowd feedbacks is now a very common practice for many web services.
% The first example that comes in mind is the search engine suggestions, like Google Autocomplete\footnote{\url{https://support.google.com/websearch/answer/106230?hl=en}}.
% Similarly, many services propose a recommendation feature centered on such feedback loops \textit{e.g.} TripAdvisor\footnote{\url{http://www.tripadvisor.com/}}, Yelp\footnote{\url{http://www.yelp.com}}.
% But there exist many other examples making use of this network effect \textit{e.g.}
% AirBnB\footnote{\url{https://www.airbnb.com}},
% Hotel Tonight\footnote{\url{https://www.hoteltonight.com/}},
% Uber\footnote{\url{https://www.uber.com/}},
% Lyft\footnote{\url{https://www.lyft.com/}},
% Home Joy\footnote{\url{https://www.homejoy.com/}},
% TaskRabbit\footnote{\url{https://www.taskrabbit.com}},
% handy\footnote{\url{https://www.handy.com/}},
% Shyp\footnote{\url{http://www.shyp.com/}}.