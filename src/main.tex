\documentclass{utils/acm_proc_article-sp}

\usepackage[utf8]{inputenc}
\usepackage[T1]{fontenc}
\usepackage{textcomp}
\usepackage[english, french]{babel}
\usepackage{algorithm}
\usepackage{algpseudocode}
\usepackage{graphicx}
\usepackage{hyperref}
\usepackage{syntax}
\usepackage{tabularx}
\usepackage{multirow}
\usepackage{hhline}

\usepackage{tikz}
\usepackage{pgfplots}
\usepgfplotslibrary{external} 
\tikzexternalize
\tikzsetexternalprefix{figures/}
\newcommand*\circled[1]{\tikz[baseline=(char.base)]{
            \node[shape=circle,draw,inner sep=0.8pt] (char) {#1};}}

\usepackage[hyperref=true,%
            url=false,%
            isbn=false,%
            style=numeric,%
            maxcitenames=3,%
            maxbibnames=100,%
            block=none]{biblatex}

\bibliography{../src/bib/OS.bib}
\bibliography{../src/bib/Flow-Based Programming.bib}
\bibliography{../src/bib/Functional Programming-Functional Reactive Programming.bib}
\bibliography{../src/bib/Stream.bib}
\bibliography{../src/bib/Dataflow.bib}
\bibliography{../src/bib/Web & Social Networks.bib}
\bibliography{../src/bib/Others.bib}
\bibliography{../src/bib/Actor Model.bib}
\bibliography{../src/bib/Misc.bib}
\bibliography{../src/bib/Parallelisation.bib}
\bibliography{../src/bib/Others.bib}
\bibliography{../src/bib/Distributed Systems.bib}
\bibliography{../src/bib/Compilation.bib}
\bibliography{../src/bib/TAJS.bib}
\bibliography{../src/bib/Javascript-static analysis.bib}

% \bibliographystyle{abbrv}
% \bibliography{../src/bib/OS.bib,../src/bib/Flow-Based Programming.bib,../src/bib/Functional Programming-Functional Reactive Programming.bib,../src/bib/Stream.bib,../src/bib/Dataflow.bib,../src/bib/Web & Social Networks.bib,../src/bib/Others.bib,../src/bib/Actor model.bib,../src/bib/Misc.bib,../src/bib/Parallelisation.bib,../src/bib/Others.bib,../src/bib/Distributed Systems.bib}

\usepackage{marginnote}
\usepackage{xcolor}

\definecolor{todo}{rgb}{0.9,0.5,0.5}
\definecolor{text}{gray}{0.8}

\newcommand{\TODO}[1]{%
	% \marginpar
	{
		\textcolor{todo}{\bf TODO}
		\textcolor{text}{#1}
	}
}

\newcommand{\ind }{%
  \hspace{4ex}%
}

\newcommand{\comment}[1]{%
  \textcolor{text}{#1}%
}

\newcommand{\ftnt}[1]{
  \footnote{\small{\url{#1}}}
}
\usepackage{color}
\usepackage{listings}
\usepackage{inconsolata}

\definecolor{red}{rgb}{1,0,0.29}
\definecolor{green}{rgb}{0,0.91,0.75}
\definecolor{lightgray}{rgb}{.9,.9,.9}
\definecolor{gray}{rgb}{.6,.6,.6}
\definecolor{darkgray}{rgb}{.4,.4,.4}

\newcommand{\testbf}{bx}

\newcommand{\CodeSymbol}[1]{
    \bfseries\textcolor{red}{#1}
}

% \definecolor{colKeys}{rgb}{0,0,1}
% \definecolor{colIdentifier}{rgb}{0,0,0}
% \definecolor{colComments}{rgb}{0,0.5,1}
% \definecolor{colString}{rgb}{0.6,0.1,0.1}



% \newcommand{\nClass}[1]{{\color{bleugris}{\textsl{\textbf{#1}}}}}
% \newcommand{\nParameter}[1]{{\color{gray}{\textbf{#1}}}}
% \newcommand{\nMethod}[1]{{\color{gray}{\textbf{#1}}}}
% \newcommand{\nConstant}[1]{\texttt{\uppercase{#1}}}
% \newcommand{\nKeyword}[1]{\textsl{\textbf{#1}}}

\newcommand\opstyle{\color{red}}

\lstdefinelanguage{js}{
  keywords={break, case, catch, continue, debugger, default, delete, do, else, false, finally, for, function, if, in, instanceof, new, null, return, switch, this, throw, true, try, typeof, var, void, while, with},
  morecomment=[l]{//},
  morecomment=[s]{/*}{*/},
  morestring=[b]',
  morestring=[b]",
  ndkeywords={class, export, boolean, throw, implements, import, this},
  keywordstyle=\color{red}\bfseries,
  ndkeywordstyle=\color{red}\bfseries,
  identifierstyle=\color{black},
  commentstyle=\color{lightgray}\ttfamily,
  stringstyle=\color{gray}\bfseries,
  basicstyle=\color{gray}\ttfamily\scriptsize,
  sensitive=true,
  escapeinside={\/\/\@}{\@},
}

\lstdefinelanguage{flx}{
  keywords={flx, fluxion},
  morecomment=[l]{//},
  morecomment=[s]{/*}{*/},
  morestring=[b]',
  morestring=[b]",
  literate= {+>}{{\ttfamily\scriptsize>>}}2
            {->}{{\CodeSymbol{->}}}2
            {>>}{{\CodeSymbol{>>}}}2, % This is a hack to get -> and >> be highlighted like keywords.
  ndkeywords={source\_js, handler\_1000, reply\_1001, null},
  keywordstyle=\color{red}\bfseries,
  ndkeywordstyle=\color{black}\bfseries,
  identifierstyle=\color{black},
  commentstyle=\color{lightgray}\ttfamily,
  stringstyle=\color{gray}\bfseries,
  basicstyle=\color{gray}\ttfamily\scriptsize,
  sensitive=true
}

\newcommand{\userlstset}[1]{
  \lstset{ %
    numberstyle=\tiny,               % the size of the fonts that are used for the line-numbers
    numbers=left,                    % where to put the line-numbers
    stepnumber=1,                    % the step between two line-numbers. If it is 1 each line will be numbered
    numbersep=5pt,                   % how far the line-numbers are from the code
    showspaces=false,                % show spaces adding particular underscores
    showstringspaces=false,          % underline spaces within strings
    showtabs=false,                  % show tabs within strings adding particular underscores
    tabsize=2,                       % sets default tabsize to 2 spaces
    captionpos=b,                    % sets the caption-position to bottom
    breaklines=true,                 % sets automatic line breaking
    breakautoindent = true,          %
    breakatwhitespace=false,         % sets if automatic breaks should only happen at whitespace
    escapeinside={\@}{\@},           % if you want to add a comment within your code
    language=#1,                     % choose the language of the code
  } %
}

\newcommand{\ic}[1]{\lstinline|#1|}

\lstnewenvironment{code}[1][js]{%
  \userlstset{#1}%
}{%
}

\newcommand{\includecode}[2]{%
  \userlstset{#1}%
  \lstinputlisting{#2}%
}
% \input{utils/alias}


% Document starts
\begin{document}

% Title portion
\title{
  % Finding equivalence between monolithic and flow based programming models
	% A fluxionnal programming langage
  % Automatic transformation from callback imbrication to sequence
  Toward automatic update from callbacks to Promises
  % Automatically update your callbacks towards Promises
}

\numberofauthors{2}
\author{
\alignauthor
Etienne Brodu, Stéphane Frénot\\
  \email{\textsf{\normalsize{\textit{firstname}.\textit{lastname}@insa-lyon.fr}}}\\
  \affaddr{\textsf{\small{Université de Lyon, INRIA,}}}\\
  \affaddr{\textsf{\small{INSA-Lyon, CITI-INRIA, F-69621, Villeurbanne, France}}}\\
\alignauthor
Frédéric Oblé\\
  \email{\textsf{\normalsize{frederic.oble@worldline.com}}}\\
  \affaddr{\textsf{\small{Worldline}}}\\
  \affaddr{\textsf{\small{53 avenue Paul Krüger - CS 60195}}}\\
  \affaddr{\textsf{\small{69624 Villeurbanne Cedex}}}
}

\maketitle

\section*{Abstract}

Javascript is the prevalent scripting language for the web.
It lets web pages register callbacks to react to user events.
A callback is a function to be invoked later with a result currently unavailable.
This pattern also proved to respond efficiently to remote requests.
Javascript is currently used to implement complete web applications.
However, callbacks are ill-suited to arrange a large asynchronous execution flow.
\textit{Promises} are a more adapted alternative.
They provide a unified control over both the synchronous and asynchronous execution flows.

The next version of Javascript proposes to replace callbacks with Promises.
% Promises are about to replace callbacks.
This paper brings the first step toward a compiler to help developers prepare this shift.
We present an equivalence between callbacks and Dues.
The latter are a simpler specification of Promises developed for the purpose of this demonstration.
From this equivalence, we implement a compiler to transform an imbrication of callbacks into a chain of Dues.
This equivalence is limited to \textit{Node.js}-style asynchronous callbacks defined \textit{in situ}.
We test our compiler over 64 \textit{npm} packages and show our results.
9 of them are compatible and compile successfully.

We consider this shift to be a first step toward the merge of concepts from the data-flow programming model into the imperative programming model.

% \vfill\eject


% A category with the (minimum) three required fields
% \category{H.4}{Information Systems Applications}{Miscellaneous}
%A category including the fourth, optional field follows...
% \category{D.2.8}{Software Engineering}{Metrics}[complexity measures, performance measures]

\category{D.3.4}{Software Engineering}{Processors}[Code generation, Compilers, Run-time environments]

\terms{Compilation}

\keywords{Flow programming, Web, Javascript} % NOT required for Proceedings
% TODO Flow programming might not be the best keywords

\vfill\eject

\newtoggle{plan}
% \toggletrue{plan}

\section{Introduction}

\iftoggle{plan}{
  20 columns papers :

  \begin{center}
    \begin{tabular}{ll}

    Abstract     \dotfill & 1 column \\
    Introduction \dotfill & 2 columns \vspace{2mm}\\

    source       \dotfill & 3 columns \\
    target       \dotfill & 3 columns \\
    equivalence  \dotfill & 4 columns \\
    test         \dotfill & 4 columns \vspace{2mm}\\

    Related Work \dotfill & 2 columns \\
    Conclusion   \dotfill & 1 columns \\

    \end{tabular}
  \end{center}
}

The world wide web started as a document sharing platform for academics.
It is now a rich application platform, pervasive, and accessible almost everywhere.
This transformation began in Netscape 2.0 with the introduction of Javascript, a web scripting language.

Javascript was originally designed for the manipulation of a graphical environment : the Document Object Model (DOM\ftnt{http://www.w3.org/DOM/}).
Functions are first class-citizens ; it allows to manipulate them like any object, and to link them to react to asynchronous events, \textit{e.g.} user inputs and remote requests.
These asynchronously triggered functions are named callbacks, and allow to efficiently cope with the distributed and inherently asynchronous architecture of the Internet.
This made Javascript a language of choice to develop both client and, more recently, server applications for the web.

Callbacks are well suited for small interactive scripts.
But in a complete application, they are ill-suited to control the larger asynchronous execution flow.
Their use leads to intricate imbrications of function calls and callbacks, commonly presented as \textit{callback hell}\ftnt{http://maxogden.github.io/callback-hell/}, or \textit{pyramid of Doom}.
This is widely recognized as a bad practice and reflects the unsuitability of callbacks in complete applications.
Eventually, developers enhanced callbacks to meet their needs with the concept of Promise\cite{Liskov1988}.

Promises bring a different way to control the asynchronous execution flow, better suited for large applications.
They fulfill this task well enough to be part of the next version of the Javascript language.
However, because Javascript started as a scripting language, beginners are often not introduced to Promises early enough, and start their code with the classical Javascript callback approach.
Moreover, despite its benefits, the concept of Promise is not yet widely acknowledged.
Developers may implement their own library for asynchronous flow control before discovering existing ones.%, like Promises.
There is such a disparity between the needs for and the acknowledgment of Promises, that there is almost 40 different implementations\ftnt{https://github.com/promises-aplus/promises-spec/blob/master/implementations.md}.
% TODO reformulate this sentence.

With the coming introduction of Promise as a language feature, we expect an increase of interest, and believe that many developers will shift to this better practice.
In this paper, we propose a compiler to automate this shift in existing code bases.
We present the transformation from an imbrication of callbacks to a sequence of Promise operations, while preserving the semantic.

Promises bring a better way to control the asynchronous execution flow, but they also impose a conditional control over the result of the execution.
Callbacks, on the other hand, leave this conditional control to the developer.
This paper focuses on the transformation from imbrication of callbacks to chain of Promises.
To avoid unnecessary modifications on this conditional control, we introduce an alternative to Promises leaving this conditional control to the developer, like callbacks.
We call this alternative specification Dues.
Our approach enables us to compile legacy Javascript code and brings a first automated step toward full Promises integration.
This simple and pragmatic compiler has been tested over \textit{65} \textit{npm} packages, \textit{10} of which with success.

In section \ref{section:definitions} we define callbacks, Promises and then Dues.
In section \ref{section:equivalence}, we explain the transformation from imbrications of callbacks to sequences of Dues.
In section \ref{section:compiler}, we present a compiler to automate the application of this equivalence.
In section \ref{section:evaluation}, we evaluate the developed compiler.
In section \ref{section:related}, we present related works, and finally conclude in section \ref{section:conclusion}.
\section{Definitions} \label{section:definitions}

\subsection{Callbacks} \label{section:definitions:continuation}

A callback is a function passed a as a parameter to a function call.
It is invoked by the callee to continue the execution with arguments not available in the caller context.
We distinguish three kinds of callbacks.

\begin{itemize}
  \item \textbf{Iterators} are functions called for each item in a set, often synchronously.
  \item \textbf{Listeners} are functions called asynchronously for each message in a stream.
  \item \textbf{Continuations} are functions called asynchronously once a result is available.
\end{itemize}

As we will see later, Promises are designed as placeholder for a unique outcome.
Iterators and Listeners are invoked multiple times resulting in multiple outcomes.
Only continuations are equivalent to Promises, or Dues.
So, we focus on continuations in this paper.

Callbacks are often mistaken for continuations ; callbacks are not inherently asynchronous, while continuations are.
In a synchronous paradigm, the sequentiality of the execution flow is trivial.
An operation needs to complete before executing the next one.
On the other hand, in an asynchronous paradigm, parallelism is trivial, operations are executed in parallel.
The sequentiality of operations needs to be explicit.
Continuations provide this control over \textbf{the sequentiality of the asynchronous execution flow}.

A continuation is a function passed as an argument to allow the callee not to block the caller until its completion.
The continuation is invoked later, at the termination of the callee to process the result as soon as possible and continue the execution ; hence the name continuation.
The continuation approach is the functional way of addressing asynchronous call without external synchronization mechanism such as IPC.

When using continuation, the convention on how to handle the result must be common for both the callee and the caller.
In \textit{Node.js}, the signature of a continuation uses the \textit{error-first}\ftnt{https://docs.nodejitsu.com/articles/errors/what-are-the-error-conventions}\ftnt{http://programmers.stackexchange.com/questions/144089/different-callbacks-for-error-or-error-as-first-argument} convention.
The first argument contains an error or \texttt{null} if no error occurred ; then follows the result.
Listing \ref{lst:continuation} is a pattern of such a continuation.
However, continuations don't inherently impose any convention.
For example, In the browser, the major convention used for continuation is the\textit{error-first} convention.

\begin{code}[js, %
             caption={Example of a continuation}, %
             label={lst:continuation}] %
my_fn(input, function continuation(error, result) {
  if (!error) {
    console.log(result);
  } else {
    throw error;
  }
});
\end{code}

% The continuation allows to continue the execution sequentially, after the result of \textit{my_fn} is available. 
% When continuations are defined inside the call, like \textit{continuation}, the sequence of deferred execution results in an intricate imbrication of calls and continuations, like in listing \ref{lst:cbhell}.
The callback hell occurs when many asynchronous calls are arranged to be executed sequentially.
Each consecutive operation adds an indentation level, because it is nested inside the continuation of the previous operation.
% That is when each caller must wait for the result before calling the next function.
It produce an imbrication of calls and function definitions, like in listing \ref{lst:cbhell}.
Because of this nesting, we say that continuations lack the \textbf{chained composition} of multiple asynchronous operations.
Promise allows to arrange such a sequence of asynchronous operations in a more readable way.


\begin{code}[js, %
             caption={Example of a sequence of continuations}, %
             label={lst:cbhell}] %
my_fn_1(input, function cont(error, result) {
  if (!error) {
    my_fn_2(result, function cont(error, result) {
      if (!error) {
        my_fn_3(result, function cont(error, result) {
          if (!error) {
            console.log(result);
          } else {
            throw error;
          }
        });
      } else {
        throw error;
      }
    });
  } else {
    throw error;
  }
});
\end{code}

\subsection{Promises} \label{section:definitions:promise}

% TODO insert these :
% Promise also provide few methods to enhance the asynchronous control flow tools\footnote{\texttt{all} and \texttt{race}}.
% There is, in Javascript, numerous Promise implementations\footnote{37 different implementations in Javascript \url{https://github.com/promises-aplus/promises-spec/blob/master/implementations.md}}.

% This section is based on the Promises section of the specification in ECMAScript 6 Harmony\ftnt{https://people.mozilla.org/~jorendorff/es6-draft.html\#sec-promise-objects} and the Promises page on the Mozilla Developer Network\ftnt{https://developer.mozilla.org/en/docs/Web/JavaScript/Reference/Global_Objects/Promise}.

The specification\ftnt{https://people.mozilla.org/~jorendorff/es6-draft.html\#sec-promise-objects} defines a promise as an object that is used as a placeholder for the eventual outcome of a deferred (and possibly asynchronous) computation.
In a synchronous paradigm, the sequentiality of the execution flow is trivial.
While in an asynchronous paradigm, this control is provided by continuations.
A Promise is an object returned by a function to represent its result, this result being synchronously or asynchronously available.
Promises provide a unified \textbf{control over the execution flow} for both paradigms.
They expose a \texttt{then} method to define the continuation to execute with the - possibly asynchronous - result.

% However, unlike continuations, the Promises specification imposes a convention on how to handle the result.
% Because it is possibly unavailable synchronously, it still requires a continuation to defer the execution when the result is made available.
% A promise requires two continuations to defer the execution in case of errors.
% These two continuations are passed to the \texttt{then} method of the promise, like illustrated in listing \ref{lst:then}.

Promises include another control over the execution flow.
They call a different function according to the outcome of the operation, one to continue the execution with the result, or the other to handle errors.
This \textbf{conditional execution} is indivisible from the Promise structure.
On the other hand, classic continuations leave this conditional execution to the developer.
As a result of this difference, Promises and continuations use two different conventions to handle errors and results.
The two conventions are illustrated in listings \ref{lst:continuation} and \ref{lst:then}.

\begin{code}[js, %
             caption={Example of a promise}, %
             label={lst:then}] %
var promise = my_fn(input)

promise.then(function onSuccess(result) {
  console.log(result);
}, function onErrors(error) {
  throw error;
});
\end{code}

As explained in section \ref{section:definitions:continuation}, continuations lack the \textbf{chained composition} of multiple asynchronous operations
Promises are specified as to arrange successions of asynchronous operations as a chain of continuations, by opposition to the imbrication of continuations illustrated in listing \ref{lst:cbhell}.
The \texttt{then} method synchronously returns a Promise linked with the Promise asynchronously returned by its continuation.
This link allow to compose \textbf{chains} of asynchronous operations.
That is to arrange them, one operation after the other, in the same indentation level.
The Promises syntax is more concise, also more readable because it is closer to the familiar synchronous paradigm.

The listing \ref{lst:promises-sequence} illustrates this chained composition.
The functions \texttt{my_fn_2} and \texttt{my_fn_3} return promises when they are executed, asynchronously.
Because these promises are not available synchronously, the method \texttt{then} returns intermediary Promises.
The latter resolve only when the former resolve.
This behavior allows to arrange the continuations in a flat chain of calls, instead of an imbrication of calls and continuations.

\begin{code}[js, %
             caption={The chain of Promises is more concise than an imbrication of callbacks}, %
             label={lst:promises-sequence}] %
my_fn_1(input)
.then(my_fn_2, onError)
.then(my_fn_3, onError)
.then(console.log, onError);

function onError(error) {
  throw error;
}
\end{code}

\subsection{Analysis} \label{seciton:definitions:analysis}

In a synchronous paradigm, the sequentiality of the execution flow is trivial.
An operation needs to complete before executing the next one.
On the other hand, in an asynchronous paradigm, parallelism is trivial, while this sequentiality needs to be explicit.
Promises and continuations provide this control over \textbf{the sequentiality of the asynchronous execution flow}.
It allows to explicitly arrange the execution of asynchronous operations one after the other, and declare a relation of causality between two operations.

As explained in section \ref{section:definitions:continuation}, continuations are invoked to hand back the result and continue the execution at the end of an asynchronous operation.
To arrange a sequence of asynchronous operations with continuations, they are nested one in the continuation of the previous, as illustrated in listing \ref{lst:cbhell}.
When the continuation is a function declared \textit{in situ}, each asynchronous operation adds a nesting level.
Because of this nesting, we say that continuations lack the \textbf{chained composition} of multiple asynchronous operations.

As illustrated in listing \ref{lst:promises-sequence}, Promises provides this chained composition.
As detailed in section \ref{section:definitions:promise}, the \texttt{then} method synchronously returns a Promise linked with the Promise asynchronously returned by its continuation.
This link allow to compose \textbf{chains} of asynchronous operations.
That is to arrange them, one operation after the other, in the same indentation level.
The Promises syntax is more readable, because it is closer to the familiar synchronous paradigm.

However, Promises include another control over the execution flow.
They call a different function according to the outcome of the asynchronous operation, one to continue the execution with the result, or the other to handle errors.
This \textbf{conditional execution} is indivisible from the Promise structure.
On the other hand, classic continuations leave this conditional execution to the developer.
As a result of this difference, Promises and continuations use two different conventions to handle errors and results.
The two conventions are illustrated in listings \ref{lst:continuation} and \ref{lst:then}.

We focus on the transformation of \textbf{the sequentiality of the execution flow}, but not on the extraction of the conditional execution.
We introduce in section \ref{section:due} a new specification, Dues.
They bring the same chained composition than Promises, while leaving the conditional execution to the developer, like continuations.

% This difference would imply a compiler to isolate the control inside the continuation.
% Such an isolation might be achieved by the compiler using a static analysis, such as the an abstract interpretation\cite{Hankin1999}.
% But this task is out of scope for this paper.
% Indeed, it is irrelevant to the transformation from imbrication to sequence and it is too complex to be explained here.



% The former uses two callbacks, one for the result and one for the errors ; while the latter uses only one, with the \textit{error-first} convention.




% \subsubsection{Specification}

% At its creation, the promise expects a callback containing the deferred computation.
% This callback is called with two functions as arguments, \texttt{resolve} to fulfill, and \texttt{reject} to reject the promise after the computation.
% % \textbf{$\warning$} The function \texttt{resolve} does \textbf{not} resolve the promise, it fulfills it.
% After its creation, the promise exposes a \texttt{then} method expecting a callback to continue the execution after the deferred computation.

% Any Promise object is in one of three mutually exclusive states: fulfilled, rejected, and pending.

% A promise \texttt{p} is fulfilled when the function \texttt{resolve} is called.
% A call to \texttt{p.then(onFulfill, onReject)} immediately call the function \texttt{onFulfill}.
% A promise \texttt{p} is rejected when the function \texttt{reject} is called.
% A call to \texttt{p.then(onFulfill, onReject)} immediately call the function \texttt{onReject}.
% A promise is pending if it is neither fulfilled nor rejected.
% A promise is settled if it is not pending, \textit{i.e.} if it is either fulfilled or rejected.
% A promise is resolved if it is settled or if it has been locked in to match the state of another promise.
% Attempting to resolve or reject a resolved promise has no effect.
% A promise is unresolved if it is not resolved.
% An unresolved promise is always in the pending state.
% A resolved promise may be pending, fulfilled or rejected.

% The \texttt{Promise} object exposes these methods :
% \begin{description}
% \item[\texttt{Promise.all(iterable)}] Returns a promise that resolves when all of the promises in the iterable argument have resolved.
% \item[\texttt{Promise.prototype.catch(onRejected)}] Appends a rejection handler callback to the promise, and returns a new promise resolving to the return value of the callback if it is called, or to its original fulfillment value if the promise is instead fulfilled.
% \item[\texttt{Promise.prototype.then(onFulfilled, onRejected)}]~\\ Appends fulfillment and rejection handlers to the promise, and returns a new promise resolving to the return value of the called handler. 
% \item[\texttt{Promise.race(iterable)}] Returns a promise that resolves or rejects as soon as one of the promises in the iterable resolves or rejects, with the value or reason from that promise.
% \item[\texttt{Promise.reject(reason)}] Returns a Promise object that is rejected with the given reason.
% \item[\texttt{Promise.resolve(value)}] Returns a Promise object that is resolved with the given value.
% If the value is a \textit{thenable}, \textit{i.e.} has a method \texttt{then}, the returned promise will follow that \textit{thenable}, adopting its eventual state; otherwise the returned promise will be fulfilled with the value.
% \end{description}

% We present in section \ref{section:spimpl} a simple implementation of Promise in Javascript.
% We only implement \texttt{then}, \texttt{resolve} and \texttt{reject} to keep the implementation concise.
%  % as they are the only methods we use for this equivalence.
% The method \texttt{catch} is redundant with the method \texttt{then}.
% The implementation for the methods \texttt{all} and \texttt{race} are out of scope in this paper.
% However, we present equivalences for both in section \ref{section:all-race}.


\section{Dues} \label{section:due}

We present an alternative to Promises called \textit{Due}.
Like Promises, a Due is an object that is used as a placeholder for the eventual outcome of a deferred computation.
Unlike Promises, and like continuations, Dues leave to the developer the control of the conditional execution over the result.
While a promise expects two continuations, \texttt{onSuccess} and \texttt{onErrors}, the method \texttt{then} of a due expects only one continuation, following the convention \textit{error-first}.
% \footnotemark{\ref{ftn:error-conventions}}
% \footnotemark{\ref{ftn:error-first}}.

A Due object is in one of two mutually exclusive states: settled or pending.
At its creation, the due expects a callback containing the deferred computation.
This callback is called synchronously with the function \texttt{settle} as argument.
The latter is invoked, potentially asynchronously, to settle the due.
Dues expose a \texttt{then} method expecting a continuation to continue the execution after its settlement.
To allow chained composition, the method \texttt{then} returns a Due linked with the due returned by its continuations.
The definition of \texttt{my\_fn} function, in listing \ref{lst:my-fn} illustrate the creation of two Dues, with synchronous and asynchronous deferred computation.

\begin{code}[js, %
             caption={Example of a due}, %
             label={lst:due}] %
var due = my_fn(input)

due.then(function continuation(error, result) {
  if (!error) {
    console.log(result);
  } else {
    throw error;
  }
});
\end{code}

% In listing \ref{lst:due}, \texttt{due} is settled when the function \texttt{settle} is called.
If \texttt{due} is settled, a call to \texttt{due.then(onSettlement)} immediately call the function \texttt{onSettlement}.
A due is pending if it is not settled.
A due is resolved if it is settled or if it has been linked with another due.
Attempting to settle a resolved due has no effect.
A resolved due may be pending or settled, while an unresolved due is always in the pending state.
The \texttt{Due} object only exposes the \texttt{then} method.
% \textbf{\texttt{Due.prototype.then(onSettlement)}}\\
% Appends settlement handlers to the due, and returns a new due resolving to the return value of the called handler.
% If the value is a \textit{thenable}, \textit{i.e.} has a method \texttt{then}, the returned due will follow that \textit{thenable}, adopting its eventual state; otherwise the returned due will be fulfilled with the value.
We present in appendix \ref{section:dueimpl} a simple implementation of Due in Javascript.

\begin{code}[js, %
             caption={Dues are chained like Promises}, %
             label={lst:dues-sequence}] %
my_fn_1(input)
.then(screenError(my_fn_2))
.then(screenError(my_fn_3))
.then(screenError(console.log));

function screenError(fn) {
  return function(error, result) {
    if (!error) {
      return fn(result);
    } else {
      throw error;
    }
  };
}
\end{code}
\section{Equivalence} \label{section:equivalence}

We present the transformation from a nested imbrication of continuations into a chain of Dues.
% The transformation must preserve the semantic.
We explain the three limitations imposed by the compiler for this transformation to preserve the semantic.
The limitations preserves the execution order, the execution linearity and the scopes of the variables used in the operations.

\subsection{Execution order}

The compiler spots function calls with a continuation, similar to the abstraction in (\ref{eq:order:source}).
It wraps the function $fn$ into the function $fn_\textbf{due}$ to return a Due.
ANd it relocates the continuation in a call to the method $\textbf{then}$, which references the Due previously returned.
The result should be similar to (\ref{eq:order:target}).
\begin{equation} \label{eq:order:source}
fn([arguments], continuation)
\end{equation}
\begin{equation} \label{eq:order:target}
fn_\textbf{due}([arguments])\textbf{.then}(continuation)
\end{equation}

The execution order is different whether $continuation$ is called synchronously, or asynchronously.
If $fn$ is synchronous, it calls the $continuation$ within its execution.
It might execute statements after the execution of $continuation$.
If $fn$ is asynchronous, the continuation is called after the end of the current execution, after $fn$.
The transformation erases this difference in the execution order.
In both cases, the transformation relocates the execution of $continuation$ after the execution of $fn$.
The latter must be asynchronous to preserve the execution order.

\subsection{Execution linearity}

The compiler transform a nested imbrication of continuations, similar to the abstraction in (\ref{eq:state:source}) into a flatten chain of calls encapsulating them, like in (\ref{eq:state:target}).
\begin{align} \label{eq:state:source}
&fn1([arguments], cont1 \{\nonumber\\
&\qquad  declare ~ variable \leftarrow result\nonumber\\
&\qquad  fn2([arguments], cont2 \{\nonumber\\
&\qquad\qquad    print ~ variable\nonumber\\
&\qquad  \})\nonumber\\
&\})
\end{align}
\begin{align} \label{eq:state:target}
&\textbf{declare variable}\nonumber\\
&fn1_\textbf{due}([arguments])\nonumber\\
&\textbf{.then}(cont1\{\nonumber\\
&\qquad  variable \leftarrow result\nonumber\\
&\qquad  fn2_\textbf{due}([arguments])\nonumber\\
&\})\nonumber\\
&\textbf{.then}(cont2\{\nonumber\\
&\qquad  print ~ variable\nonumber\\
&\})
\end{align}

An imbrication of continuations must not contain any loop, nor function definition that is not a continuation.
Both modify the linearity of the execution flow which is required for the equivalence to keep the semantic.
A call nested inside a loop returns multiple Dues, while only one is returned to continue the chain.
And a call nested inside a function definition is unable to return the Due to continue the chain.
On the other hand, conditional branching leaves the execution linearity and the semantic intact.
If the nested asynchronous function is not called, the execution of the chain stops as expected.

\subsection{Variable scope}

In (\ref{eq:state:source}), the definitions of $cont1$ and $cont2$ are overlapping.
The $variable$ declared in $cont1$ is accessible in $cont2$ to be printed.
In (\ref{eq:state:target}), however, definitions of $cont1$ and $cont2$ are not overlapping, they are siblings.
The $variable$ is not accessible to $cont2$, it must be relocated in a parent function to be accessible by both $cont1$ and $cont2$.
To detect such variables, the compiler must infer their scope statically.
Languages with a lexical scope define the scope of a variable statically.
That is most imperative languages, like C/C++, Python, Ruby or Java.
The subset of Javascript excluding the built-in functions \texttt{with} and \texttt{eval} is also lexically scoped.
To compile Javascript, the compiler must exclude programs using these two statements.
\section{Compiler} \label{section:compiler}

We build a compiler to automate the application of this equivalence on existing Javascript repositories.
The compilation process contains two important steps, the identification of the continuations, and the generation of chains.

\subsection{Continuation identification}

The equivalence is applicable only on continuations.
The first compilation step is to identify the continuations among the callbacks.
A continuation is a callback invoked only once, asynchronously.
Spotting a continuation implies to identify these two conditions.
There is no syntactical difference between a synchronous and an asynchronous callee.
And it is impossible to assure a callback to be invoked only once, because the implementation of the callee is often unavailable statically.
Therefore, the identification of continuations is necessarily based on semantical differences.
For this purpose, the compiler would need to have a deep understanding of the control and data flows of the program.
Because of the highly dynamic nature of Javascript, this understanding is either unsound, limited, or complex.
Instead, we choose to leave to the developer the identification of compatible continuations among the identified callbacks.
They are expected to understand the limitations of this compiler, and the semantic of the code to compile.

We provide a simple interface for developers to interact with the compiler.
We built the compiler as a web page.
The compiler is available online\ftnt{compiler-due.apps.zone52.org} to reproduce the tests.

This interaction prevent the complete automation of the individual compilation process.
However, we are working on an automation at a global scale.
We expect to be able to identify of a continuation only based on its callee name, \textit{e.g.} \texttt{fs.readFile}.
We built a service to gather these names along with their identification.
The compiler query this service to present an estimated identification, and send back the final choices to refine it.
In future works, we would like to study the possibility of easing the compilation interaction.

\subsection{Chains}

The compositions of continuations and Dues are arranged differently.
Continuations structure the execution flow as a tree, while the chain of Dues arrange it sequentially.
A parent continuation can have several children, not a Due.
The second compilation step is to identify these imbrications of continuations to transform them into chains.

% We consider an imbrication of continuations as a subtree of the linear execution tree.
The compiler trims each tree of continuations to get chains to translate into Dues.
If a continuation has more than one child, the compiler try to find a legitimate child to continue the chain.
The legitimate child is the only parent among its siblings.
If there is several parents among the children, then none are the legitimate child.
The non legitimate children start a new tree.
This steps yield chains of continuations assured to be transformable into sequences of Dues.
The code generation from these chains is straightforward.






% \comment{TODO insert this}
% We developed the compiler core in node.js Javascript.
% There already exist sets of tools for manipulating code in Javascript.
% We used the Esprima suite of tools.


\section{Evaluation} \label{section:evaluation}

To validate our compiler, we used it to compile several Javascript projects likely to contains continuations.
We present the results of these tests.

% \subsection{Projects selection}

The compilation of a project require user interaction.
To conduct the test in a reasonable time, we limit the test set to a minimum.
We search the \textit{Node Package Manager} database to restrict the set to \textit{Node.js} projects.
We refine the selection to web applications depending on the web framework \textit{express}, but not on the most common asynchronous libraries such as \textit{Q} and \textit{Async}.
We refine further the selection to projects using the test frameworks \textit{mocha} in its default configuration.
The test set contains 65 projects.
This subset is very small, and cannot represent the wide possibilities of Javascript.
However, we believe it is sufficient to represent a majority of common cases.

% \subsection{Run the tests}

For each project, we verify that the project is correctly tested, and pass the tests.
During the compilation, the user identifies the compatible continuations among the detected callbacks.
We test the compilation result, and verify that the test result is the same that the original tests.
The compilation result should pass the tests as well.
This is not a strong validation, but it assure the compiler to work in the most common cases.

% \subsection{Results}

On the 65 projects tested, almost a majority, 29 (45\%), does not contain any compatible continuations.
We reckon that these projects use continuations the compiler didn't detect.
10 (15\%) projects are not compilable because they contain \texttt{with} or \texttt{eval} statements.
5 (8\%) projects used less common asynchronous libraries we didn't filter previously.
4 (6\%) projects are not syntactically correct.
4 (6\%) projects fail their tests before the compilation.
3 (5\%) projects are not tested.
And finally, 10 (15\%) projects compiled successfully.
The compiler did not fail to compile any projects.

Over the successfully compiled projects, the compiler detected 172 callbacks.
We manually identified 56 of them to be compatible continuations.
One project contains 20 continuations, the others contains between 9 and 1 continuations each.
On the 56 continuations, 36 are single, and 20 continuations involved in an imbrication.
There are 5 imbrications containing 2 continuations, 2 imbrication containing 3 continuations, and one imbrication containing 4 continuations.
The result of these tests prove the compiler to be able to successfully transform imbrications of continuations.



























% Tested applications :

% rest-api-express            -> OK

% socket-testing              -> OK (partial) the project is broken, but it seems broken in the same way before and after the compilation :)
% nodeExample                 -> OK (partial) read-simple.js and write-simple.js compile successfully (the only two files to contain continuation). but the project compiler breaks, I suspect it is because of some !# stuffs. 

% fligg                       -> NOK not functionnal : impossible to log, or sign in
% colors                      -> NOK no continuations
% request                     -> NOK no continuations=:
% async                       -> NOK presence of with or eval
% penguin                     -> NOK not functionnal : syntax error
% expression.io               -> NOK tests not passed before compilation

% blog-experiment             -> NOK already use thenable



% NPM

% app-json-fetcher              -> NOK no continuations
% express-resource-plus         -> NOK no continuations
% fizzbuzz-hypermedia-server    -> NOK no continuations
% flair                         -> NOK no continuations
% generator-wikismith           -> NOK no continuations
% brokowski                     -> NOK no continuations
% claus                         -> NOK no continuations
% costa                         -> NOK no continuations
% express-device                -> NOK no continuations
% heroku-proxy                  -> NOK no continuations
% http-test-servers             -> NOK no continuations
% jellyjs-plugin-httpserver     -> NOK no continuations
% loopback-angular-cli          -> NOK no continuations
% loopback-explorer             -> NOK no continuations
% monami                        -> NOK no continuations
% mongoose-epxress              -> NOK no continuations
% nodebootstrap-server          -> NOK no continuations
% oauth-express                 -> NOK no continuations
% public-server                 -> NOK no continuations
% scrapit                       -> NOK no continuations
% sik                           -> NOK no continuations
% sonea                         -> NOK no continuations (coffee)
% vsoft-explorer                -> NOK no continuations
% webs-weeia                    -> NOK no continuations
% code-connect-server           -> NOK no continuations
% csp-endpoint                  -> NOK no continuations
% glsl-transition-minify        -> NOK no continuations
% moby                          -> NOK no continuations
% squirrel-server               -> NOK no continuations
%   >> 29 no continuations

% arkhaios                      -> NOK presence of with or eval
% browserman                    -> NOK presence of with or eval
% infectwit                     -> NOK presence of with or eval
% ldapp                         -> NOK presence of with or eval
% levelhud                      -> NOK presence of with or eval
% manet                         -> NOK presence of with or eval
% solid                         -> NOK presence of with or eval
% swac                          -> NOK presence of with or eval
% swac-odm                      -> NOK presence of with or eval
% adnoce                        -> NOK presence of with or eval
%   >> 10 with or eval

% boomerang-server              -> NOK already use thenable (es6-promise)
% hyper.io                      -> NOK already use thenable (when)
% node-lanetix-microservice     -> NOK already use thenable (bluebird)
% librarian                     -> NOK already use thenable (bluebird)
% webtasks                      -> NOK already use async stuffs (subtask)
%   >> 5 async frameworks

% xtc                           -> NOK unexpected token < use templating engine to generate js@
% agenda-ui                     -> NOK unexpected reserved word use ES6 specifications
% traffic-light                 -> NOK illegal return statement
% ellipsis                      -> NOK illegal return statement
%   >> 4 incorrect code

% express-orm-mvc               -> NOK test failed (this project is just a template)
% hangout                       -> NOK test failed (unfinished ?)
% derp                          -> NOK test failed + this project use generators
% ord                           -> NOK however test failed similarly after and before compilation
%   >> 4 test failed

% tuzi                          -> NOK no tests
% inchi-server                  -> NOK no tests
% otwo                          -> NOK no tests
%   >> 3 no tests

% to compile
% express-couchUser             >> 13 chains (20 continuations / 40)   1: 9,      2: 2,      3: 1,      4: 1
% gifsockets-server             >> 1 chains (1 continuation / 3)       1: 1
% node-heroku-bouncer           >> 3 chains (3 continuations / 7)      1: 3
% moonridge ( +install morgan)  >> 4 chains (6 continuations / 37)     1: 2      2: 2
% redis-key-overview            >> 7 chains (9 continuations / 14)     1: 6                  3: 1
% slack-integrator              >> 2 chains (3 continuations / 6)      1: 1      2: 1
% timbits                       >> 8 chains (8 continuations / 34)     1: 8
% tingo-rest                    >> 4 chains (4 continuations / 12)     1: 4
% express-endpoint              >> 2 chains (2 continuations / 19)     1: 2 and already use async stuffs (step)
%   >> 10 OK total of 56 continuations / 172, 1:36 2:5 3:2 4:1


% 29 no continuations
% 10 with ot eval
% 5 async
% 4 incorrect code
% 4 test failed
% 3 no tests
% 10 OK

% total 65 Projects
\section{Related works}

Promise

crowd sourced compilation
\section{Conclusion} \label{section:conclusion}

In this paper, we introduced a compiler to automatically transform an imbrication of continuations into a sequence.
Firstly, we defined callbacks and Promise as the base for this work.
We then introduced Due, a new specification similar to Promise, to carry the demonstration of this transformation.
We presented the equivalence between a continuation and a Due, and the composition of this equivalence for imbricated continuations.
And finally, we presented a compiler to automate this transformation on actual code bases.

A continuation share its scope with its descendance, \textit{i.e.} the following imbricated continuations.
While A callback due can not share its own identifiers with its descendance, \textit{i.e.} the following dues.
Their scope are disjoints.
However, it can share global identifiers, and object references.
This difference of accessibility imposes the compiler to segment the asynchronous control flow.
This segmentation is soft : the stack is independent, but the heap is shared.

The callback of a Due returns another due, for the asynchronous operation completions to trigger the next.
The result of an asynchronous operation is passed to the next through the Due - or Promise - structure.
A serie of asynchronous operations operated by Dues - or Promises - is very suggestive of a data flow process.
It is a chain of operations feeding the next with the result of the previous.

We aim at pushing further this analogy.
We want to impose the compiler to bring complete independance to asynchronous operations.
So that the only communication is of their results along the flow.
Such a compiler would be able to transform a monolithic program into a chain of independent asynchronous operations linked by a flow of data.
We expect the possibility for new execution models to take advantage of this independence to bring performance scalability.
While developers continue using the monolithic model for its evolution scalability.

\vfill\eject

\printbibliography[]
\vfill\eject
\appendix

\section{Due implementation} \label{section:dueimpl}

We present the implementation of Due in listing \ref{lst:simplepromise}, with a small set of test cases in listing \ref{lst:testpromise}.

\includecode{js, %
             caption={Implementation of Due}, %
             label={lst:due}
             }
             {../due/src/index.js}


\includecode{js, %
             caption={Tests for the implementation of Due}, %
             label={lst:testdue}
             }
             {../due/test/index.js}

% \section{Simple Promise implementation} \label{section:spimpl}

% We present a simple implementation of Promise in listing \ref{lst:simplepromise}, with a small set of test cases in listing \ref{lst:testpromise}.

% \includecode{js, %
%              caption={Simple implementation of Promise}, %
%              label={lst:simplepromise}
%              }
%              {snippets/SimplePromise/src/index.js}


% \includecode{js, %
%              caption={Tests for the simple implementation of Promise}, %
%              label={lst:testpromise}
%              }
%              {snippets/SimplePromise/test/index.js}



\end{document}