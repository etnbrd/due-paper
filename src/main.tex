\documentclass{utils/acm_proc_article-sp}

\usepackage[utf8]{inputenc}
\usepackage[T1]{fontenc}
\usepackage{textcomp}
\usepackage[english, french]{babel}
\usepackage[babel=true]{csquotes}
\usepackage{algorithm}
\usepackage{algpseudocode}
\usepackage{graphicx}
\usepackage{hyperref}
\usepackage{syntax}
\usepackage{arevmath}
\usepackage{multirow}

\usepackage{tikz}
\usepackage{pgfplots}
\usepgfplotslibrary{external} 
\tikzexternalize
\tikzsetexternalprefix{figures/}
\newcommand*\circled[1]{\tikz[baseline=(char.base)]{
            \node[shape=circle,draw,inner sep=0.8pt] (char) {#1};}}

\usepackage[hyperref=true,%
            url=false,%
            isbn=false,%
            style=numeric,%
            maxcitenames=3,%
            maxbibnames=100,%
            block=none]{biblatex}

\bibliography{../src/bib/OS.bib}
\bibliography{../src/bib/Flow-Based Programming.bib}
\bibliography{../src/bib/Functional Programming-Functional Reactive Programming.bib}
\bibliography{../src/bib/Stream.bib}
\bibliography{../src/bib/Dataflow.bib}
\bibliography{../src/bib/Web & Social Networks.bib}
\bibliography{../src/bib/Others.bib}
\bibliography{../src/bib/Actor Model.bib}
\bibliography{../src/bib/Misc.bib}
\bibliography{../src/bib/Parallelisation.bib}
\bibliography{../src/bib/Others.bib}
\bibliography{../src/bib/Distributed Systems.bib}
\bibliography{../src/bib/Compilation.bib}
\bibliography{../src/bib/TAJS.bib}

% \bibliographystyle{abbrv}
% \bibliography{../src/bib/OS.bib,../src/bib/Flow-Based Programming.bib,../src/bib/Functional Programming-Functional Reactive Programming.bib,../src/bib/Stream.bib,../src/bib/Dataflow.bib,../src/bib/Web & Social Networks.bib,../src/bib/Others.bib,../src/bib/Actor model.bib,../src/bib/Misc.bib,../src/bib/Parallelisation.bib,../src/bib/Others.bib,../src/bib/Distributed Systems.bib}

\usepackage{marginnote}
\usepackage{xcolor}

\definecolor{todo}{rgb}{0.9,0.5,0.5}
\definecolor{text}{gray}{0.8}

\newcommand{\TODO}[1]{%
	% \marginpar
	{
		\textcolor{todo}{\bf TODO}
		\textcolor{text}{#1}
	}
}

\newcommand{\ind }{%
  \hspace{4ex}%
}

\newcommand{\comment}[1]{%
  \textcolor{text}{#1}%
}

\newcommand{\ftnt}[1]{
  \footnote{\small{\url{#1}}}
}
\usepackage{color}
\usepackage{listings}
\usepackage{inconsolata}

\definecolor{red}{rgb}{1,0,0.29}
\definecolor{green}{rgb}{0,0.91,0.75}
\definecolor{lightgray}{rgb}{.9,.9,.9}
\definecolor{gray}{rgb}{.6,.6,.6}
\definecolor{darkgray}{rgb}{.4,.4,.4}

\newcommand{\testbf}{bx}

\newcommand{\CodeSymbol}[1]{
    \bfseries\textcolor{red}{#1}
}

% \definecolor{colKeys}{rgb}{0,0,1}
% \definecolor{colIdentifier}{rgb}{0,0,0}
% \definecolor{colComments}{rgb}{0,0.5,1}
% \definecolor{colString}{rgb}{0.6,0.1,0.1}



% \newcommand{\nClass}[1]{{\color{bleugris}{\textsl{\textbf{#1}}}}}
% \newcommand{\nParameter}[1]{{\color{gray}{\textbf{#1}}}}
% \newcommand{\nMethod}[1]{{\color{gray}{\textbf{#1}}}}
% \newcommand{\nConstant}[1]{\texttt{\uppercase{#1}}}
% \newcommand{\nKeyword}[1]{\textsl{\textbf{#1}}}

\newcommand\opstyle{\color{red}}

\lstdefinelanguage{js}{
  keywords={break, case, catch, continue, debugger, default, delete, do, else, false, finally, for, function, if, in, instanceof, new, null, return, switch, this, throw, true, try, typeof, var, void, while, with},
  morecomment=[l]{//},
  morecomment=[s]{/*}{*/},
  morestring=[b]',
  morestring=[b]",
  ndkeywords={class, export, boolean, throw, implements, import, this},
  keywordstyle=\color{red}\bfseries,
  ndkeywordstyle=\color{red}\bfseries,
  identifierstyle=\color{black},
  commentstyle=\color{lightgray}\ttfamily,
  stringstyle=\color{gray}\bfseries,
  basicstyle=\color{gray}\ttfamily\scriptsize,
  sensitive=true,
  escapeinside={\/\/\@}{\@},
}

\lstdefinelanguage{flx}{
  keywords={flx, fluxion},
  morecomment=[l]{//},
  morecomment=[s]{/*}{*/},
  morestring=[b]',
  morestring=[b]",
  literate= {+>}{{\ttfamily\scriptsize>>}}2
            {->}{{\CodeSymbol{->}}}2
            {>>}{{\CodeSymbol{>>}}}2, % This is a hack to get -> and >> be highlighted like keywords.
  ndkeywords={source\_js, handler\_1000, reply\_1001, null},
  keywordstyle=\color{red}\bfseries,
  ndkeywordstyle=\color{black}\bfseries,
  identifierstyle=\color{black},
  commentstyle=\color{lightgray}\ttfamily,
  stringstyle=\color{gray}\bfseries,
  basicstyle=\color{gray}\ttfamily\scriptsize,
  sensitive=true
}

\newcommand{\userlstset}[1]{
  \lstset{ %
    numberstyle=\tiny,               % the size of the fonts that are used for the line-numbers
    numbers=left,                    % where to put the line-numbers
    stepnumber=1,                    % the step between two line-numbers. If it is 1 each line will be numbered
    numbersep=5pt,                   % how far the line-numbers are from the code
    showspaces=false,                % show spaces adding particular underscores
    showstringspaces=false,          % underline spaces within strings
    showtabs=false,                  % show tabs within strings adding particular underscores
    tabsize=2,                       % sets default tabsize to 2 spaces
    captionpos=b,                    % sets the caption-position to bottom
    breaklines=true,                 % sets automatic line breaking
    breakautoindent = true,          %
    breakatwhitespace=false,         % sets if automatic breaks should only happen at whitespace
    escapeinside={\@}{\@},           % if you want to add a comment within your code
    language=#1,                     % choose the language of the code
  } %
}

\newcommand{\ic}[1]{\lstinline|#1|}

\lstnewenvironment{code}[1][js]{%
  \userlstset{#1}%
}{%
}

\newcommand{\includecode}[2]{%
  \userlstset{#1}%
  \lstinputlisting{#2}%
}
% \input{utils/alias}


% Document starts
\begin{document}

% Title portion
\title{
  % Finding equivalence between monolithic and flow based programming models
	% A fluxionnal programming langage
  Compilation from callback imbrication to sequence
}

\numberofauthors{2}
\author{
\alignauthor
Etienne Brodu, Stéphane Frénot\\
  \email{\textsf{\normalsize{\textit{firstname}.\textit{lastname}@insa-lyon.fr}}}\\
  \affaddr{\textsf{\small{Université de Lyon, INRIA,}}}\\
  \affaddr{\textsf{\small{INSA-Lyon, CITI-INRIA, F-69621, Villeurbanne, France}}}\\
\alignauthor
Fabien Cellier, Frédéric Oblé\\
  \email{\textsf{\normalsize{\textit{firstname}.\textit{lastname}@worldline.com}}}\\
  \affaddr{\textsf{\small{Worldline}}}\\
  \affaddr{\textsf{\small{53 avenue Paul Krüger - CS 60195}}}\\
  \affaddr{\textsf{\small{69624 Villeurbanne Cedex}}}
}

\maketitle

% \section*{Abstract}

Javascript is the prevalent scripting language for the web.
It lets web pages register callbacks to react to user events.
A callback is a function to be invoked later with a result currently unavailable.
This pattern also proved to respond efficiently to remote requests.
Javascript is currently used to implement complete web applications.
However, callbacks are ill-suited to arrange a large asynchronous execution flow.
\textit{Promises} are a more adapted alternative.
They provide a unified control over both the synchronous and asynchronous execution flows.

The next version of Javascript proposes to replace callbacks with Promises.
% Promises are about to replace callbacks.
This paper brings the first step toward a compiler to help developers prepare this shift.
We present an equivalence between callbacks and Dues.
The latter are a simpler specification of Promises developed for the purpose of this demonstration.
From this equivalence, we implement a compiler to transform an imbrication of callbacks into a chain of Dues.
This equivalence is limited to \textit{Node.js}-style asynchronous callbacks defined \textit{in situ}.
We test our compiler over 64 \textit{npm} packages and show our results.
9 of them are compatible and compile successfully.

We consider this shift to be a first step toward the merge of concepts from the data-flow programming model into the imperative programming model.

% \vfill\eject


% A category with the (minimum) three required fields
% \category{H.4}{Information Systems Applications}{Miscellaneous}
%A category including the fourth, optional field follows...
% \category{D.2.8}{Software Engineering}{Metrics}[complexity measures, performance measures]

\category{D.3.4}{Software Engineering}{Processors}[Code generation, Compilers, Run-time environments]

\terms{Compilation}

\keywords{Flow programming, Web, Javascript} % NOT required for Proceedings
% TODO Flow programming might not be the best keywords

\newtoggle{plan}
% \toggletrue{plan}

\section*{Abstract}

Javascript is the prevalent scripting language for the web.
It lets web pages register callbacks to react to user events.
A callback is a function to be invoked later with a result currently unavailable.
This pattern also proved to respond efficiently to remote requests.
Javascript is currently used to implement complete web applications.
However, callbacks are ill-suited to arrange a large asynchronous execution flow.
\textit{Promises} are a more adapted alternative.
They provide a unified control over both the synchronous and asynchronous execution flows.

The next version of Javascript proposes to replace callbacks with Promises.
% Promises are about to replace callbacks.
This paper brings the first step toward a compiler to help developers prepare this shift.
We present an equivalence between callbacks and Dues.
The latter are a simpler specification of Promises developed for the purpose of this demonstration.
From this equivalence, we implement a compiler to transform an imbrication of callbacks into a chain of Dues.
This equivalence is limited to \textit{Node.js}-style asynchronous callbacks defined \textit{in situ}.
We test our compiler over 64 \textit{npm} packages and show our results.
9 of them are compatible and compile successfully.

We consider this shift to be a first step toward the merge of concepts from the data-flow programming model into the imperative programming model.

% \vfill\eject
\section{Introduction}

\iftoggle{plan}{
  20 columns papers :

  \begin{center}
    \begin{tabular}{ll}

    Abstract     \dotfill & 1 column \\
    Introduction \dotfill & 2 columns \vspace{2mm}\\

    source       \dotfill & 3 columns \\
    target       \dotfill & 3 columns \\
    equivalence  \dotfill & 4 columns \\
    test         \dotfill & 4 columns \vspace{2mm}\\

    Related Work \dotfill & 2 columns \\
    Conclusion   \dotfill & 1 columns \\

    \end{tabular}
  \end{center}
}

The world wide web started as a document sharing platform for academics.
It is now a rich application platform, pervasive, and accessible almost everywhere.
This transformation began in Netscape 2.0 with the introduction of Javascript, a web scripting language.

Javascript was originally designed for the manipulation of a graphical environment : the Document Object Model (DOM\ftnt{http://www.w3.org/DOM/}).
Functions are first class-citizens ; it allows to manipulate them like any object, and to link them to react to asynchronous events, \textit{e.g.} user inputs and remote requests.
These asynchronously triggered functions are named callbacks, and allow to efficiently cope with the distributed and inherently asynchronous architecture of the Internet.
This made Javascript a language of choice to develop both client and, more recently, server applications for the web.

Callbacks are well suited for small interactive scripts.
But in a complete application, they are ill-suited to control the larger asynchronous execution flow.
Their use leads to intricate imbrications of function calls and callbacks, commonly presented as \textit{callback hell}\ftnt{http://maxogden.github.io/callback-hell/}, or \textit{pyramid of Doom}.
This is widely recognized as a bad practice and reflects the unsuitability of callbacks in complete applications.
Eventually, developers enhanced callbacks to meet their needs with the concept of Promise\cite{Liskov1988}.

Promises bring a different way to control the asynchronous execution flow, better suited for large applications.
They fulfill this task well enough to be part of the next version of the Javascript language.
However, because Javascript started as a scripting language, beginners are often not introduced to Promises early enough, and start their code with the classical Javascript callback approach.
Moreover, despite its benefits, the concept of Promise is not yet widely acknowledged.
Developers may implement their own library for asynchronous flow control before discovering existing ones.%, like Promises.
There is such a disparity between the needs for and the acknowledgment of Promises, that there is almost 40 different implementations\ftnt{https://github.com/promises-aplus/promises-spec/blob/master/implementations.md}.
% TODO reformulate this sentence.

With the coming introduction of Promise as a language feature, we expect an increase of interest, and believe that many developers will shift to this better practice.
In this paper, we propose a compiler to automate this shift in existing code bases.
We present the transformation from an imbrication of callbacks to a sequence of Promise operations, while preserving the semantic.

Promises bring a better way to control the asynchronous execution flow, but they also impose a conditional control over the result of the execution.
Callbacks, on the other hand, leave this conditional control to the developer.
This paper focuses on the transformation from imbrication of callbacks to chain of Promises.
To avoid unnecessary modifications on this conditional control, we introduce an alternative to Promises leaving this conditional control to the developer, like callbacks.
We call this alternative specification Dues.
Our approach enables us to compile legacy Javascript code and brings a first automated step toward full Promises integration.
This simple and pragmatic compiler has been tested over \textit{65} \textit{npm} packages, \textit{10} of which with success.

In section \ref{section:definitions} we define callbacks, Promises and then Dues.
In section \ref{section:equivalence}, we explain the transformation from imbrications of callbacks to sequences of Dues.
In section \ref{section:compiler}, we present a compiler to automate the application of this equivalence.
In section \ref{section:evaluation}, we evaluate the developed compiler.
In section \ref{section:related}, we present related works, and finally conclude in section \ref{section:conclusion}.
\section{Definitions} \label{section:definitions}

\subsection{Callbacks} \label{section:definitions:continuation}

A callback is a function passed a as a parameter to a function call.
It is invoked by the callee to continue the execution with arguments not available in the caller context.
We distinguish three kinds of callbacks.

\begin{itemize}
  \item \textbf{Iterators} are functions called for each item in a set, often synchronously.
  \item \textbf{Listeners} are functions called asynchronously for each message in a stream.
  \item \textbf{Continuations} are functions called asynchronously once a result is available.
\end{itemize}

As we will see later, Promises are designed as placeholder for a unique outcome.
Iterators and Listeners are invoked multiple times resulting in multiple outcomes.
Only continuations are equivalent to Promises, or Dues.
So, we focus on continuations in this paper.

Callbacks are often mistaken for continuations ; callbacks are not inherently asynchronous, while continuations are.
In a synchronous paradigm, the sequentiality of the execution flow is trivial.
An operation needs to complete before executing the next one.
On the other hand, in an asynchronous paradigm, parallelism is trivial, operations are executed in parallel.
The sequentiality of operations needs to be explicit.
Continuations provide this control over \textbf{the sequentiality of the asynchronous execution flow}.

A continuation is a function passed as an argument to allow the callee not to block the caller until its completion.
The continuation is invoked later, at the termination of the callee to process the result as soon as possible and continue the execution ; hence the name continuation.
The continuation approach is the functional way of addressing asynchronous call without external synchronization mechanism such as IPC.

When using continuation, the convention on how to handle the result must be common for both the callee and the caller.
In \textit{Node.js}, the signature of a continuation uses the \textit{error-first}\ftnt{https://docs.nodejitsu.com/articles/errors/what-are-the-error-conventions}\ftnt{http://programmers.stackexchange.com/questions/144089/different-callbacks-for-error-or-error-as-first-argument} convention.
The first argument contains an error or \texttt{null} if no error occurred ; then follows the result.
Listing \ref{lst:continuation} is a pattern of such a continuation.
However, continuations don't inherently impose any convention.
For example, In the browser, the major convention used for continuation is the\textit{error-first} convention.

\begin{code}[js, %
             caption={Example of a continuation}, %
             label={lst:continuation}] %
my_fn(input, function continuation(error, result) {
  if (!error) {
    console.log(result);
  } else {
    throw error;
  }
});
\end{code}

% The continuation allows to continue the execution sequentially, after the result of \textit{my_fn} is available. 
% When continuations are defined inside the call, like \textit{continuation}, the sequence of deferred execution results in an intricate imbrication of calls and continuations, like in listing \ref{lst:cbhell}.
The callback hell occurs when many asynchronous calls are arranged to be executed sequentially.
Each consecutive operation adds an indentation level, because it is nested inside the continuation of the previous operation.
% That is when each caller must wait for the result before calling the next function.
It produce an imbrication of calls and function definitions, like in listing \ref{lst:cbhell}.
Because of this nesting, we say that continuations lack the \textbf{chained composition} of multiple asynchronous operations.
Promise allows to arrange such a sequence of asynchronous operations in a more readable way.


\begin{code}[js, %
             caption={Example of a sequence of continuations}, %
             label={lst:cbhell}] %
my_fn_1(input, function cont(error, result) {
  if (!error) {
    my_fn_2(result, function cont(error, result) {
      if (!error) {
        my_fn_3(result, function cont(error, result) {
          if (!error) {
            console.log(result);
          } else {
            throw error;
          }
        });
      } else {
        throw error;
      }
    });
  } else {
    throw error;
  }
});
\end{code}

\subsection{Promises} \label{section:definitions:promise}

% TODO insert these :
% Promise also provide few methods to enhance the asynchronous control flow tools\footnote{\texttt{all} and \texttt{race}}.
% There is, in Javascript, numerous Promise implementations\footnote{37 different implementations in Javascript \url{https://github.com/promises-aplus/promises-spec/blob/master/implementations.md}}.

% This section is based on the Promises section of the specification in ECMAScript 6 Harmony\ftnt{https://people.mozilla.org/~jorendorff/es6-draft.html\#sec-promise-objects} and the Promises page on the Mozilla Developer Network\ftnt{https://developer.mozilla.org/en/docs/Web/JavaScript/Reference/Global_Objects/Promise}.

The specification\ftnt{https://people.mozilla.org/~jorendorff/es6-draft.html\#sec-promise-objects} defines a promise as an object that is used as a placeholder for the eventual outcome of a deferred (and possibly asynchronous) computation.
In a synchronous paradigm, the sequentiality of the execution flow is trivial.
While in an asynchronous paradigm, this control is provided by continuations.
A Promise is an object returned by a function to represent its result, this result being synchronously or asynchronously available.
Promises provide a unified \textbf{control over the execution flow} for both paradigms.
They expose a \texttt{then} method to define the continuation to execute with the - possibly asynchronous - result.

% However, unlike continuations, the Promises specification imposes a convention on how to handle the result.
% Because it is possibly unavailable synchronously, it still requires a continuation to defer the execution when the result is made available.
% A promise requires two continuations to defer the execution in case of errors.
% These two continuations are passed to the \texttt{then} method of the promise, like illustrated in listing \ref{lst:then}.

Promises include another control over the execution flow.
They call a different function according to the outcome of the operation, one to continue the execution with the result, or the other to handle errors.
This \textbf{conditional execution} is indivisible from the Promise structure.
On the other hand, classic continuations leave this conditional execution to the developer.
As a result of this difference, Promises and continuations use two different conventions to handle errors and results.
The two conventions are illustrated in listings \ref{lst:continuation} and \ref{lst:then}.

\begin{code}[js, %
             caption={Example of a promise}, %
             label={lst:then}] %
var promise = my_fn(input)

promise.then(function onSuccess(result) {
  console.log(result);
}, function onErrors(error) {
  throw error;
});
\end{code}

As explained in section \ref{section:definitions:continuation}, continuations lack the \textbf{chained composition} of multiple asynchronous operations
Promises are specified as to arrange successions of asynchronous operations as a chain of continuations, by opposition to the imbrication of continuations illustrated in listing \ref{lst:cbhell}.
The \texttt{then} method synchronously returns a Promise linked with the Promise asynchronously returned by its continuation.
This link allow to compose \textbf{chains} of asynchronous operations.
That is to arrange them, one operation after the other, in the same indentation level.
The Promises syntax is more concise, also more readable because it is closer to the familiar synchronous paradigm.

The listing \ref{lst:promises-sequence} illustrates this chained composition.
The functions \texttt{my_fn_2} and \texttt{my_fn_3} return promises when they are executed, asynchronously.
Because these promises are not available synchronously, the method \texttt{then} returns intermediary Promises.
The latter resolve only when the former resolve.
This behavior allows to arrange the continuations in a flat chain of calls, instead of an imbrication of calls and continuations.

\begin{code}[js, %
             caption={The chain of Promises is more concise than an imbrication of callbacks}, %
             label={lst:promises-sequence}] %
my_fn_1(input)
.then(my_fn_2, onError)
.then(my_fn_3, onError)
.then(console.log, onError);

function onError(error) {
  throw error;
}
\end{code}

\subsection{Analysis} \label{seciton:definitions:analysis}

In a synchronous paradigm, the sequentiality of the execution flow is trivial.
An operation needs to complete before executing the next one.
On the other hand, in an asynchronous paradigm, parallelism is trivial, while this sequentiality needs to be explicit.
Promises and continuations provide this control over \textbf{the sequentiality of the asynchronous execution flow}.
It allows to explicitly arrange the execution of asynchronous operations one after the other, and declare a relation of causality between two operations.

As explained in section \ref{section:definitions:continuation}, continuations are invoked to hand back the result and continue the execution at the end of an asynchronous operation.
To arrange a sequence of asynchronous operations with continuations, they are nested one in the continuation of the previous, as illustrated in listing \ref{lst:cbhell}.
When the continuation is a function declared \textit{in situ}, each asynchronous operation adds a nesting level.
Because of this nesting, we say that continuations lack the \textbf{chained composition} of multiple asynchronous operations.

As illustrated in listing \ref{lst:promises-sequence}, Promises provides this chained composition.
As detailed in section \ref{section:definitions:promise}, the \texttt{then} method synchronously returns a Promise linked with the Promise asynchronously returned by its continuation.
This link allow to compose \textbf{chains} of asynchronous operations.
That is to arrange them, one operation after the other, in the same indentation level.
The Promises syntax is more readable, because it is closer to the familiar synchronous paradigm.

However, Promises include another control over the execution flow.
They call a different function according to the outcome of the asynchronous operation, one to continue the execution with the result, or the other to handle errors.
This \textbf{conditional execution} is indivisible from the Promise structure.
On the other hand, classic continuations leave this conditional execution to the developer.
As a result of this difference, Promises and continuations use two different conventions to handle errors and results.
The two conventions are illustrated in listings \ref{lst:continuation} and \ref{lst:then}.

We focus on the transformation of \textbf{the sequentiality of the execution flow}, but not on the extraction of the conditional execution.
We introduce in section \ref{section:due} a new specification, Dues.
They bring the same chained composition than Promises, while leaving the conditional execution to the developer, like continuations.

% This difference would imply a compiler to isolate the control inside the continuation.
% Such an isolation might be achieved by the compiler using a static analysis, such as the an abstract interpretation\cite{Hankin1999}.
% But this task is out of scope for this paper.
% Indeed, it is irrelevant to the transformation from imbrication to sequence and it is too complex to be explained here.



% The former uses two callbacks, one for the result and one for the errors ; while the latter uses only one, with the \textit{error-first} convention.




% \subsubsection{Specification}

% At its creation, the promise expects a callback containing the deferred computation.
% This callback is called with two functions as arguments, \texttt{resolve} to fulfill, and \texttt{reject} to reject the promise after the computation.
% % \textbf{$\warning$} The function \texttt{resolve} does \textbf{not} resolve the promise, it fulfills it.
% After its creation, the promise exposes a \texttt{then} method expecting a callback to continue the execution after the deferred computation.

% Any Promise object is in one of three mutually exclusive states: fulfilled, rejected, and pending.

% A promise \texttt{p} is fulfilled when the function \texttt{resolve} is called.
% A call to \texttt{p.then(onFulfill, onReject)} immediately call the function \texttt{onFulfill}.
% A promise \texttt{p} is rejected when the function \texttt{reject} is called.
% A call to \texttt{p.then(onFulfill, onReject)} immediately call the function \texttt{onReject}.
% A promise is pending if it is neither fulfilled nor rejected.
% A promise is settled if it is not pending, \textit{i.e.} if it is either fulfilled or rejected.
% A promise is resolved if it is settled or if it has been locked in to match the state of another promise.
% Attempting to resolve or reject a resolved promise has no effect.
% A promise is unresolved if it is not resolved.
% An unresolved promise is always in the pending state.
% A resolved promise may be pending, fulfilled or rejected.

% The \texttt{Promise} object exposes these methods :
% \begin{description}
% \item[\texttt{Promise.all(iterable)}] Returns a promise that resolves when all of the promises in the iterable argument have resolved.
% \item[\texttt{Promise.prototype.catch(onRejected)}] Appends a rejection handler callback to the promise, and returns a new promise resolving to the return value of the callback if it is called, or to its original fulfillment value if the promise is instead fulfilled.
% \item[\texttt{Promise.prototype.then(onFulfilled, onRejected)}]~\\ Appends fulfillment and rejection handlers to the promise, and returns a new promise resolving to the return value of the called handler. 
% \item[\texttt{Promise.race(iterable)}] Returns a promise that resolves or rejects as soon as one of the promises in the iterable resolves or rejects, with the value or reason from that promise.
% \item[\texttt{Promise.reject(reason)}] Returns a Promise object that is rejected with the given reason.
% \item[\texttt{Promise.resolve(value)}] Returns a Promise object that is resolved with the given value.
% If the value is a \textit{thenable}, \textit{i.e.} has a method \texttt{then}, the returned promise will follow that \textit{thenable}, adopting its eventual state; otherwise the returned promise will be fulfilled with the value.
% \end{description}

% We present in section \ref{section:spimpl} a simple implementation of Promise in Javascript.
% We only implement \texttt{then}, \texttt{resolve} and \texttt{reject} to keep the implementation concise.
%  % as they are the only methods we use for this equivalence.
% The method \texttt{catch} is redundant with the method \texttt{then}.
% The implementation for the methods \texttt{all} and \texttt{race} are out of scope in this paper.
% However, we present equivalences for both in section \ref{section:all-race}.


\section{Dues} \label{section:due}

We present an alternative to Promises called \textit{Due}.
Like Promises, a Due is an object that is used as a placeholder for the eventual outcome of a deferred computation.
Unlike Promises, and like continuations, Dues leave to the developer the control of the conditional execution over the result.
While a promise expects two continuations, \texttt{onSuccess} and \texttt{onErrors}, the method \texttt{then} of a due expects only one continuation, following the convention \textit{error-first}.
% \footnotemark{\ref{ftn:error-conventions}}
% \footnotemark{\ref{ftn:error-first}}.

A Due object is in one of two mutually exclusive states: settled or pending.
At its creation, the due expects a callback containing the deferred computation.
This callback is called synchronously with the function \texttt{settle} as argument.
The latter is invoked, potentially asynchronously, to settle the due.
Dues expose a \texttt{then} method expecting a continuation to continue the execution after its settlement.
To allow chained composition, the method \texttt{then} returns a Due linked with the due returned by its continuations.
The definition of \texttt{my\_fn} function, in listing \ref{lst:my-fn} illustrate the creation of two Dues, with synchronous and asynchronous deferred computation.

\begin{code}[js, %
             caption={Example of a due}, %
             label={lst:due}] %
var due = my_fn(input)

due.then(function continuation(error, result) {
  if (!error) {
    console.log(result);
  } else {
    throw error;
  }
});
\end{code}

% In listing \ref{lst:due}, \texttt{due} is settled when the function \texttt{settle} is called.
If \texttt{due} is settled, a call to \texttt{due.then(onSettlement)} immediately call the function \texttt{onSettlement}.
A due is pending if it is not settled.
A due is resolved if it is settled or if it has been linked with another due.
Attempting to settle a resolved due has no effect.
A resolved due may be pending or settled, while an unresolved due is always in the pending state.
The \texttt{Due} object only exposes the \texttt{then} method.
% \textbf{\texttt{Due.prototype.then(onSettlement)}}\\
% Appends settlement handlers to the due, and returns a new due resolving to the return value of the called handler.
% If the value is a \textit{thenable}, \textit{i.e.} has a method \texttt{then}, the returned due will follow that \textit{thenable}, adopting its eventual state; otherwise the returned due will be fulfilled with the value.
We present in appendix \ref{section:dueimpl} a simple implementation of Due in Javascript.

\begin{code}[js, %
             caption={Dues are chained like Promises}, %
             label={lst:dues-sequence}] %
my_fn_1(input)
.then(screenError(my_fn_2))
.then(screenError(my_fn_3))
.then(screenError(console.log));

function screenError(fn) {
  return function(error, result) {
    if (!error) {
      return fn(result);
    } else {
      throw error;
    }
  };
}
\end{code}
\section{Equivalences} \label{section:equivalences}

In the previous section, we present the difference between continuation and Dues
Both allow control over the sequentiality of the execution flow.
When using only continuation, sequence of asynchronous operations are nested, one in the continuation of the next. 
On the other hand, Dues allow the linear composition of continuations.

Based on this difference, we present two examples of source code manipulation to transform continuation into Dues.
The first manipulation is the simplest one.
It transforms a unique continuation into a Due.
The second manipulation is the composition of the first manipulation.
It transforms nested continuations into a linear sequence of Dues.
This second manipulation requires to move the continuation definitions, which modifies the semantic.
We finally present a static lexical analysis to modify the source code before the manipulation to avoid the semantic modification.

The main advantage for developers to use Dues, is to flatten the overlapping continuations into a more readable, linear sequence.
The nesting of continuations only occurs when they are defined by \textit{FunctionExpressions}\footnote{\url{http://www.ecma-international.org/ecma-262/5.1/\#sec-11.2.5}}.
When the continuation is not declared \textit{in situ}, it avoids the imbrication of function declarations and calls.
We focus only on the modification of continuation declared \textit{in situ}.
Moreover, the transformation is \textit{sound} only when manipulating \textit{FunctionExpressions}, as explained in section \ref{section:equivaences:general}.
% This equivalence would not improve readability.
% Moreover, it would require heavier manipulation of the source code, as explained in section \ref{section:equivaences:general}.

The transformations presented modifies the syntax of the asynchronous call.
The asynchronous function needs to be modified to return a Due, instead of expecting a continuation.
For the demonstrations, we use the function \texttt{my_fn} in listing \ref{lst:my-fn}.
It both expects a callback and returns a Due.
There is no libraries compatible with both callback and Dues, like \texttt{my_fn}.
However, the Due library provide a function \texttt{mock} to transform a function expecting continuation into a function returning a Due.
We don't focus neither on the replacement of these libraries, nor on the detection of their methods in the source code.
We expect the continuations to be already screened out from other callbacks, either by a developer, or by another automated tool.
We address this problem in section \ref{section:compiler:lib-compilation}.

\includecode{js, %
             caption={Example of two function expecting a callback, and returning a due, one synchronous the other asynchronous.}, %
             label={lst:my-fn}}{snippets/my-fn.js}

\subsection{Simple equivalence} \label{section:equivalences:general}

As explained in section \ref{section:definitions:callback}, a continuation is a function passed as argument to defer its execution, like in listing \ref{lst:ct-ex}.
As explained in section \ref{section:due}, a Due is an object to defer a computation, and exposes a method \texttt{then} to continue the execution after the deferred computation, like in listing \ref{lst:du-ex}.

Because the difference between continuations and dues is the composition, the difference between the listings \ref{lst:ct-ex} and \ref{lst:du-ex} is mainly syntactical.
The transformation is immediate, and trivial.
% As illustrated in listing \ref{lst:my-fn}, \texttt{my_fn} both accepts a callback and returns a Due.
The manipulation consist of calling the method \texttt{then} of the Due returned by \texttt{my_fn}, and moving \texttt{continuation} to the arguments of this new call.
In Javascript, when entering a scope, declaration of variables and functions are processed before any execution.
Declaring an identifier anywhere in a scope is equivalent to declaring it at the top.
The identifier \texttt{continuation}, is declared before the call to \texttt{my_fn} in both listings \ref{lst:ct-ex} and \ref{lst:du-ex}.
This behavior is called \textit{hoisting}.
The manipulation is \textit{sound} because it conserves the semantic.% for \textit{FunctionExpression} like \texttt{continuation}.

For other types of continuations, \textit{e.g.} an expression returning a function, this manipulation modifies the execution order.
Before the manipulation, the expression evaluation would occur \textbf{before} the call to \texttt{my_fn}.
While, after the manipulation, the expression evaluation would occur \textbf{after} the call to \texttt{my_fn}.
If the expression evaluation produces expected side-effects, the manipulation would prevent them from happening before the call to \texttt{my_fn}.
The manipulation is \textit{sound} only when manipulating \textit{FunctionExpression}.

\includecode{js, %
             caption={A simple continuation}, %
             label={lst:ct-ex}
             }
             {snippets/ct-ex.js}

\includecode{js, %
             caption={A simple Due is very similar to a simple continuation}, %
             label={lst:du-ex}
             }
             {snippets/du-ex.js}

\subsection{Composition of nested continuations} \label{section:overlapping-continuations}

The previous manipulation allows the modification of only one continuation.
To transform a nested pyramid of continuations into a sequence of Dues, we need to assure the composition of this simple transformation.
An example of nested pyramid of continuation is illustrated in listing \ref{lst:ct-seq}.
The expected result for the composition is illustrated in listing \ref{lst:du-seq}.

In listing \ref{lst:ct-seq}, the two continuations definition, \texttt{ct1} line \ref{lst:ct-seq:ct1} and \texttt{ct2} line \ref{lst:ct-seq:ct2}, are overlapping.
While, in listing \ref{lst:du-seq}, they are not overlapping, they are defined sequentially, one after the other.
The transformation between \ref{lst:ct-seq} and \ref{lst:du-seq} is similar to the previous transformation, only two more transformations are required.
For the linear composition, \texttt{ct1} must \textit{a)} retrieves the Due returned by the second call to \texttt{my_fn}, line \ref{lst:du-seq:ctdef2}, and \textit{b)} returns it, line \ref{lst:du-seq:ret}.

The composition of the simpler manipulation leads to two semantical differences between listing \ref{lst:ct-seq} and \ref{lst:du-seq}.
Moving the definition of \texttt{ct2} is not \textit{sound}.
\begin{itemize}
  \item In listing \ref{lst:ct-seq}, if \texttt{my\_fn} calls \texttt{ct2} synchronously, its execution occurs before \circled{2}, line \ref{lst:ct-seq:cm2}.
  While in listing \ref{lst:du-seq}, whether the Due returned by \texttt{my\_fn} settles synchronously or not, the execution of \texttt{ct2} occurs after \circled{2}, line \ref{lst:du-seq:cm2}.
  To keep the semantic intact, only continuations of asynchronous functions can be turned into Dues.
  We need to assure the asynchronism of \texttt{my\_fn}.
  \item In listing \ref{lst:ct-seq}, because the definitions of \texttt{ct1} and \texttt{ct2} are overlapping, their environment record, commonly called scope, are also overlapping.
  The function \texttt{ct1} shares its identifiers with \texttt{ct2}.
  While in listing \ref{lst:du-seq}, the definitions of \texttt{ct1} and \texttt{ct2} are siblings, so \texttt{ct1} and \texttt{ct2} have disjoint scopes.
  If \texttt{ct2} uses identifiers defined in \texttt{ct1}, the manipulation makes them inaccessible.
  To keep the semantic intact, we need to analyze their scope to assure their disjunction before the manipulation. 
  We address this issue in section \ref{section:disjunction}.
\end{itemize}

\includecode{js, %
             caption={Overlapping continuations definitions}, %
             label={lst:ct-seq}
             }
             {snippets/ct-seq.js}

\includecode{js, %
             caption={Sequential continuations definitions using Dues}, %
             label={lst:du-seq}
             }
             {snippets/du-seq.js}

\subsection{Assure environment record disjunction} \label{section:disjunction}

In Javascript, a function defines a \textit{Lexical Environment}\footnote{\url{https://people.mozilla.org/~jorendorff/es6-draft.html\#sec-lexical-environments}}.
A \textit{Lexical Environment} defines the scope of a function.
It consists of an \textit{Environment Record} and a - potentially null - reference to an outer \textit{Lexical Environment}.
An \textit{Environment Record} records the identifier bindings that are created within the scope of its associated \textit{Lexical Environment}.
Javascript exposes two built-in functions that dynamically modify \textit{Lexical Environment} : \texttt{eval} and \texttt{with}.

To avoid dynamical modifications of Lexical Environment, we consider a subset of Javascript, excluding \texttt{eval} and \texttt{with}.
This subset is statically - or lexically - scoped at the function level.
A \textit{Lexical Environment} is static, it is immutable during run time.
It is possible to infer the identifiers and their scopes before run time.
The scope of an identifier is limited to the defining function and its children.

In listing \ref{lst:ct-seq}, the scopes of \texttt{ct1} and \texttt{ct2} are overlapping.
The \textit{Lexical Environment} of \texttt{ct1} is the outer environment of the \textit{Lexical Environment} of \texttt{ct2}.
The identifier \texttt{shared_identifier} declared line \ref{lst:ct-seq:shared-identifier}, is accessible from \texttt{ct2}.
However, in listing \ref{lst:vo-seq}, the \textit{Environment Records} of \texttt{ct1} and \texttt{ct2} are siblings.
The identifiers declared in \texttt{ct1} are no longer accessible from \texttt{ct2}.
To move the child \textit{Environment Records} out of its parent while keeping the semantic, it needs to be disjoint from its parent.
Two environment records are disjoints if they don't share any identifiers.
Two environment records are joints if they share at least one identifier.
A shared identifier is replaceable by an identifier declared in the parent outer environment record to be accessible by both the parent and the child.
The identifier \texttt{shared_identifier} is moved to the outer environment, shared by both \texttt{ct1} and \texttt{ct2}.
In listings \ref{lst:ct-seq} and \ref{lst:vo-seq} this outer environment is the global environment records.

As assured in section \ref{seciton:overlapping-callbacks}, the deferred computation is asynchronous.
And the execution flow is not modified by the manipulation.
The function \texttt{ct2} is executed after the function \texttt{ct2}, and they share the same environment record.
So all type of accesses are equivalents : writing or reading.
The type of access required by \texttt{ct1} and \texttt{ct2} is insignificant for this manipulation.



% \subsection{Soundness and Completeness} \label{section:soundness-completeness}

% \TODO{TODO prove soundness and completeness with the following}
% The call to \texttt{my_fn} is a \textit{CallExpression}\footnote{\url{https://people.mozilla.org/~jorendorff/es6-draft.html\#sec-expression-rules}}.
% The arguments of a CallExpression are only AssignementExpression.
% The AssignementExpression that possibly returns a callable object, \textit{i.e.} a function, or a method, are :
% \begin{itemize}
% \item Identifier
% \item FunctionExpression
% \item ArrowFunction
% \item YieldExpression
% \item CallExpression
% \item MemberExpression
% \item this
% \end{itemize}
% In the listings \ref{lst:cb-simple}, \ref{lst:pr-simple}, \ref{lst:ct-seq} and \ref{lst:pr-seq}, the identifier \texttt{callback} can be replaced with any of these expressions.

























\section{Compiler} \label{section:compiler}

\comment{TODO insert that}\\
We expect developers to have a limited control over the implementation of the callee.
In \textit{Node.js}, asynchronous functions are either part of the \textit{Node.js} API like \texttt{fs}, or wrapper from libraries like \texttt{express}.
Instead of modifying the implementation of the callee, we propose to wrap it inside a function which transform its signature, leaving the semantic intact.
The \texttt{due} library provides such a wrapper : \texttt{mock}.



We explain in this section the compilation process.
The compiler transform asynchronous call with continuation to make them compatible with due.
This process flatten a continuation pyramid into a cascading sequence of call to \texttt{then}.
There is roughly two steps in this process.
The first, described in section \ref{section:compiler:chain}, is to build the chain of continuation from the continuations pyramids.
The second, described in section \ref{section:compiler:chain}, is to extract the shared identifiers to move them in a parent scope.

As stated earlier, the compiler doesn't detect rupture points.
It expects a list of previously detected rupture points.
In the prototype, we spot the rupture point by hand.
In section \ref{section:compiler:lib-compilation}, we present some thoughts about automation solutions.

\subsection{Build continuation chains} \label{section:compiler:chain}

% As explained in section \ref{section:definitions}, the cascade of Due is possible because the method \texttt{then} returns a due which resolve when the promise returned by its callback resolves.
% As continuations are directly called from the event-loop, their return values are discarded.
% The return statement in a continuation is only to control the execution flow - early return - not the data flow - return a value.
% We can modify a continuation to make it return a Due while keeping the semantic.

The first step is to build arrange the rupture points in chain.
These chains are branches of trees of rupture points.

A tree of rupture points represent the hierarchy of the rupture points in the source code.
To form this tree, there is only one constraint : a child rupture point cannot be separated from its parent by a function.
This is because this middle function is not assured to be executed only once, or synchronously.
If this middle function is used as an iterator or a listener, there would be multiple child Dues to return, while only one is expected by the parent callback.
If this middle function is used as a continuation, the due returned by the child rupture point would net be available synchronously to be returned by the parent callback.
For example this middle function might be defined in the parent, but used in a different part of the program.

At the end of this first process, we have multiple trees containing the hierarchy of all the rupture points in the application.
Because a function can only return one Due, it is not possible to flatten a tree of rupture points, only a chain.
As a callback cannot return more than one Due, it is not possible to build a sequence of Due from a tree.
The next step of the compilation is to trim the trees to obtain chains of callbacks transformable into sequence of Due.

Each tree is walked to find rupture point with more than one child.
If there is more than one child, we try to find a legitimate child to continue the chain.
A legitimate child is a child with at least one child.
If there is more than one legitimate child, all are discarded, they all start new chains.
The non legitimate child start a new tree to walk the same way.

The result is a list of chains of rupture points.
Each chain is assured to be transformable into a sequence of \texttt{then} calls.
However, as stated earlier, this transformation modifies the scopes organization.
To keep the semantic intact, we need to modify the source code in some way that allow the flattening modification to keep the semantic intact.


\subsection{Identifier extraction} \label{section:compiler:extraction}

To keep the semantic intact after the flattening of rupture points, no identifier must be shared between two callbacks.
Every declaration of shared identifiers is extracted in a parent scope.

We iterate over the rupture point in a chain.
If there is any reference to a variable in the children rupture points, then this variable is marked as shared.
If the rupture point is not a parent, the descendants scope are not modified by the flattening process.

All shared variables are extracted from their current scope, and placed in the scope at the root of the chain so to be shared by all callbacks in the chain.
If there is a conflict with another variable in this root scope, it is necessary to rename one of these variables.



\subsection{Crowd sourced compilation} \label{section:compiler:lib-compilation}

Spotting rupture points is equivalent to spotting continuation from other callbacks.
A continuation is defined only by its invocation.
Spotting a continuation means identifying the function called with the continuation as argument.
Function, in Javascript, are first-class citizen, they can take many forms.
Statically identifying a function expecting a continuation implies the compiler to have a very deep understanding of the program.
This understanding comes from certain static analyses which don't guarantee a good enough result.

If it is not possible to automate the screening process at an individual scale, it might be possible to automate it at a global scale.
Most rupture point calls are expected to have distinct names, \textit{e.g.} \texttt{fs.readFile}.
In future works, we would like to study the possibility to harvest the result of every compilation to build a list of common rupture points.
With this list, it would be possible to approximate this automation to ease the compilation interaction.

% Also, safe check : warn the user when a callback is called synchronously while it shouldn't.
\section{Related works}

Promise

crowd sourced compilation
\section{Conclusion} \label{section:conclusion}

In this paper, we introduced a compiler to automatically transform an imbrication of continuations into a sequence.
Firstly, we defined callbacks and Promise as the base for this work.
We then introduced Due, a new specification similar to Promise, to carry the demonstration of this transformation.
We presented the equivalence between a continuation and a Due, and the composition of this equivalence for imbricated continuations.
And finally, we presented a compiler to automate this transformation on actual code bases.

A continuation share its scope with its descendance, \textit{i.e.} the following imbricated continuations.
While A callback due can not share its own identifiers with its descendance, \textit{i.e.} the following dues.
Their scope are disjoints.
However, it can share global identifiers, and object references.
This difference of accessibility imposes the compiler to segment the asynchronous control flow.
This segmentation is soft : the stack is independent, but the heap is shared.

The callback of a Due returns another due, for the asynchronous operation completions to trigger the next.
The result of an asynchronous operation is passed to the next through the Due - or Promise - structure.
A serie of asynchronous operations operated by Dues - or Promises - is very suggestive of a data flow process.
It is a chain of operations feeding the next with the result of the previous.

We aim at pushing further this analogy.
We want to impose the compiler to bring complete independance to asynchronous operations.
So that the only communication is of their results along the flow.
Such a compiler would be able to transform a monolithic program into a chain of independent asynchronous operations linked by a flow of data.
We expect the possibility for new execution models to take advantage of this independence to bring performance scalability.
While developers continue using the monolithic model for its evolution scalability.

\printbibliography[]
% \vfill\eject
\appendix

\section{Due implementation} \label{section:dueimpl}

We present the implementation of Due in listing \ref{lst:simplepromise}, with a small set of test cases in listing \ref{lst:testpromise}.

\includecode{js, %
             caption={Implementation of Due}, %
             label={lst:due}
             }
             {../due/src/index.js}


\includecode{js, %
             caption={Tests for the implementation of Due}, %
             label={lst:testdue}
             }
             {../due/test/index.js}

% \section{Simple Promise implementation} \label{section:spimpl}

% We present a simple implementation of Promise in listing \ref{lst:simplepromise}, with a small set of test cases in listing \ref{lst:testpromise}.

% \includecode{js, %
%              caption={Simple implementation of Promise}, %
%              label={lst:simplepromise}
%              }
%              {snippets/SimplePromise/src/index.js}


% \includecode{js, %
%              caption={Tests for the simple implementation of Promise}, %
%              label={lst:testpromise}
%              }
%              {snippets/SimplePromise/test/index.js}



\end{document}