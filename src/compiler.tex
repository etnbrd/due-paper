\section{Compiler} \label{section:compiler}

\comment{TODO insert that}\\
We expect developers to have a limited control over the implementation of the callee.
In \textit{Node.js}, asynchronous functions are either part of the \textit{Node.js} API like \texttt{fs}, or wrapper from libraries like \texttt{express}.
Instead of modifying the implementation of the callee, we propose to wrap it inside a function which transform its signature, leaving the semantic intact.
The \texttt{due} library provides such a wrapper : \texttt{mock}.


From the equivalence previously detailed, this section present an automation of the translation.
The compiler translates Javascript code with continuations into code with Promises.
There is roughly two steps in the compilation process, present in the first two following sections.
The first is to identify the continuation pyramids.
The second is to assure the disjunction of nested continuations.
In the third following section, we the identification of the rupture points.
% This identification is a combination of static analysis, manual confirmation and community reinforcement.

% This identification in the source code is done using esprima\ftnt{http://esprima.org/}, an ECMAScript parser.

% We explain in this section the compilation process.
% The compiler transform asynchronous call with continuation to make them compatible with due.
% This process flatten a continuation pyramid into a cascading sequence of call to \texttt{then}.

\subsection{Build continuation chains} \label{section:compiler:chain}

% As explained in section \ref{section:definitions}, the cascade of Due is possible because the method \texttt{then} returns a due which resolve when the promise returned by its callback resolves.
% As continuations are directly called from the event-loop, their return values are discarded.
% The return statement in a continuation is only to control the execution flow - early return - not the data flow - return a value.
% We can modify a continuation to make it return a Due while keeping the semantic.

To apply the equivalence, the compiler needs to identify the chains of nested continuations to translate into Dues.
Dues are arranged as a chain of calls, while, a continuation can nest multiple child continuations.
They are arranged as a tree.
The compiler identify these trees of continuations, and break it into chains of continuations.
As stated in the previous section, we consider only continuations that are \textit{FunctionExpressions}.

Any Javascript code is organized as a tree of function definitions nested one in its parent.
A tree of continuation is a part of this tree of function definitions containing only continuations.
A continuation is a leaf only if its function parent is also a continuation.
Otherwise, it is the root of a tree.

A tree of continuations can not contain a loop, nor a function that is not a continuation.
Indeed, a loop might arrange the execution flow to call the nested asynchronous function multiple times, which would produce multiple Dues, while only one is expected to continue the chain.
Or the function might be invoked so as to break the chain of Dues.


However, a conditional branch leaves the semantic intact.

If function is called multiple times, there would be more than one Due to return, while only one is expected to continue the chain.
If function is used as a continuation, the due returned by the child rupture point would net be available synchronously to be returned by the parent callback.
For example this middle function might be defined in the parent, but used in a different part of the program.


We consider only continuation because Dues are a placeholder for a single outcome.


At the end of this first process, we have multiple trees containing the hierarchy of all the rupture points in the application.
Because a function can only return one Due, it is not possible to flatten a tree of rupture points, only a chain.
As a callback cannot return more than one Due, it is not possible to build a sequence of Due from a tree.
The next step of the compilation is to trim the trees to obtain chains of callbacks transformable into sequence of Due.

Each tree is walked to find rupture point with more than one child.
If there is more than one child, we try to find a legitimate child to continue the chain.
A legitimate child is a child with at least one child.
If there is more than one legitimate child, all are discarded, they all start new chains.
The non legitimate child start a new tree to walk the same way.

The result is a list of chains of rupture points.
Each chain is assured to be transformable into a sequence of \texttt{then} calls.
However, as stated earlier, this transformation modifies the scopes organization.
To keep the semantic intact, we need to modify the source code in some way that allow the flattening modification to keep the semantic intact.


\subsection{Identifier extraction} \label{section:compiler:extraction}

To keep the semantic intact after the flattening of rupture points, no identifier must be shared between two callbacks.
Every declaration of shared identifiers is extracted in a parent scope.

We iterate over the rupture point in a chain.
If there is any reference to a variable in the children rupture points, then this variable is marked as shared.
If the rupture point is not a parent, the descendants scope are not modified by the flattening process.

All shared variables are extracted from their current scope, and placed in the scope at the root of the chain so to be shared by all callbacks in the chain.
If there is a conflict with another variable in this root scope, it is necessary to rename one of these variables.



\subsection{Crowd sourced compilation} \label{section:compiler:lib-compilation}

Spotting rupture points is equivalent to spotting continuation from other callbacks.
A continuation is defined only by its invocation.
Spotting a continuation means identifying the function called with the continuation as argument.
Function, in Javascript, are first-class citizen, they can take many forms.
Statically identifying a function expecting a continuation implies the compiler to have a very deep understanding of the program.
This understanding comes from certain static analyses which don't guarantee a good enough result.

If it is not possible to automate the screening process at an individual scale, it might be possible to automate it at a global scale.
Most rupture point calls are expected to have distinct names, \textit{e.g.} \texttt{fs.readFile}.
In future works, we would like to study the possibility to harvest the result of every compilation to build a list of common rupture points.
With this list, it would be possible to approximate this automation to ease the compilation interaction.

% Also, safe check : warn the user when a callback is called synchronously while it shouldn't.