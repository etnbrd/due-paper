\section{Compiler} \label{section:compiler}

We explain in this section the compilation process.
The compiler transform asynchronous call with continuation to make them compatible with due.
This process flatten a continuation pyramid into a cascading sequence of call to \texttt{then}.
There is roughly two steps in this process.
The first, described in section \ref{section:compiler:chain}, is to build the chain of continuation from the continuations pyramids.
The second, described in section \ref{section:compiler:chain}, is to extract the shared identifiers to move them in a parent scope.

As stated earlier, the compiler doesn't detect rupture points.
It expects a list of previously detected rupture points.
In the prototype, we spot the rupture point by hand.
In section \ref{section:compiler:lib-compilation}, we present some thoughts about automation solutions.

\subsection{Build continuation chains} \label{section:compiler:chain}

% As explained in section \ref{section:definitions}, the cascade of Due is possible because the method \texttt{then} returns a due which resolve when the promise returned by its callback resolves.
% As continuations are directly called from the event-loop, their return values are discarded.
% The return statement in a continuation is only to control the execution flow - early return - not the data flow - return a value.
% We can modify a continuation to make it return a Due while keeping the semantic.

The first step is to build arrange the rupture points in chain.
These chains are branches of trees of rupture points.

A tree of rupture points represent the hierarchy of the rupture points in the source code.
To form this tree, there is only one constraint : a child rupture point cannot be separated from its parent by a function.
This is because this middle function is not assured to be executed only once, or synchronously.
If this middle function is used as an iterator or a listener, there would be multiple child Dues to return, while only one is expected by the parent callback.
If this middle function is used as a continuation, the due returned by the child rupture point would net be available synchronously to be returned by the parent callback.
For example this middle function might be defined in the parent, but used in a different part of the program.

At the end of this first process, we have multiple trees containing the hierarchy of all the rupture points in the application.
Because a function can only return one Due, it is not possible to flatten a tree of rupture points, only a chain.
As a callback cannot return more than one Due, it is not possible to build a sequence of Due from a tree.
The next step of the compilation is to trim the trees to obtain chains of callbacks transformable into sequence of Due.

Each tree is walked to find rupture point with more than one child.
If there is more than one child, we try to find a legitimate child to continue the chain.
A legitimate child is a child with at least one child.
If there is more than one legitimate child, all are discarded, they all start new chains.
The non legitimate child start a new tree to walk the same way.

The result is a list of chains of rupture points.
Each chain is assured to be transformable into a sequence of \texttt{then} calls.
However, as stated earlier, this transformation modifies the scopes organization.
To keep the semantic intact, we need to modify the source code in some way that allow the flattening modification to keep the semantic intact.


\subsection{Identifier extraction} \label{section:compiler:extraction}

To keep the semantic intact after the flattening of rupture points, no identifier must be shared between two callbacks.
Every declaration of shared identifiers is extracted in a parent scope.

We iterate over the rupture point in a chain.
If there is any reference to a variable in the children rupture points, then this variable is marked as shared.
If the rupture point is not a parent, the descendants scope are not modified by the flattening process.

All shared variables are extracted from their current scope, and placed in the scope at the root of the chain so to be shared by all callbacks in the chain.
If there is a conflict with another variable in this root scope, it is necessary to rename one of these variables.



\subsection{Crowd sourced compilation} \label{section:compiler:lib-compilation}

Spotting rupture points is equivalent to spotting continuation from other callbacks.
A continuation is defined only by its invocation.
Spotting a continuation means identifying the function called with the continuation as argument.
Function, in Javascript, are first-class citizen, they can take many forms.
Statically identifying a function expecting a continuation implies the compiler to have a very deep understanding of the program.
This understanding comes from certain static analyses which don't guarantee a good enough result.

If it is not possible to automate the screening process at an individual scale, it might be possible to automate it at a global scale.
Most rupture point calls are expected to have distinct names, \textit{e.g.} \texttt{fs.readFile}.
In future works, we would like to study the possibility to harvest the result of every compilation to build a list of common rupture points.
With this list, it would be possible to approximate this automation to ease the compilation interaction.

% Also, safe check : warn the user when a callback is called synchronously while it shouldn't.