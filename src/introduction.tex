\section{Introduction}

\iftoggle{plan}{
  20 columns papers :

  \begin{center}
    \begin{tabular}{ll}

    Abstract     \dotfill & 1 column \\
    Introduction \dotfill & 2 columns \vspace{2mm}\\

    source       \dotfill & 3 columns \\
    target       \dotfill & 3 columns \\
    equivalence  \dotfill & 4 columns \\
    test         \dotfill & 4 columns \vspace{2mm}\\

    Related Work \dotfill & 2 columns \\
    Conclusion   \dotfill & 1 columns \\

    \end{tabular}
  \end{center}
}

\section{Introduction}

Why javascript is a success ?
-> success of the web

Why do we need asynchronism ?
-> event-loop and stuffs are better at scalability
-> web is very low latency, synchronism is not efficient

Why do we need to sugar syntax asynchronism ?
-> we are used to synchronism, asynchronism is like brain rape for most.

Callbacks and promises are two different tools to arrange the flow of deferred operations, possibly asynchronously.
Callbacks implies the inversion of the control.
It often result in an intricate imbrication of function definitions and calls, called the callback hell, or the pyramid of doom\footnote{\raggedright http://maxogden.github.io/callback-hell/}.
Too much impribrated callback is largely considered a bad practice.
Promises are an alternative to avoid this imbrication.
It allows to replace the overlapping callbacks by a cascading\footnote{\url{http://stackoverflow.com/questions/758486/how-to-implement-javascript-cascades}} sequence of call.
This paper presents an equivalence to transform callbacks into Promises.
To do so, we define a simpler alternative to Promise, called Vow.
We present an equivalence to transform callbacks to Vows, and then an equivalence to transform Vows into Promises.
We intend to transform the callback hell into a flatten sequence of promises.

Why Node.js ?
-> we target server side code ?
-> callback hell very present because code base are sometime heavy ?